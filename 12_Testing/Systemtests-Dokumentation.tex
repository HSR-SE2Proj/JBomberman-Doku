\documentclass[11pt]{scrartcl}

\title{Systemtests: JBomberman}
\author{Silvan Adrian \\ Fabian Binna \\ Pascal Kistler}
\date{\today{}}

\usepackage[ngerman]{babel}
\usepackage[automark]{scrpage2}
\usepackage{hyperref}
\usepackage{color}
\usepackage[normalem]{ulem}
\usepackage{scrpage2}
\usepackage{graphicx}
\usepackage{tabularx}
\graphicspath{ {./images/} }
\pagestyle{scrheadings}

\clearscrheadfoot
\ihead{\includegraphics[scale=0.4]{jbomberman}}
\ohead{Projekt: JBomberman}
\ifoot{Systemtests: JBomberman}
\cfoot{Version: 1.04}
\ofoot{Datum: 27.05.15}
\setheadsepline{0.5pt}
\setfootsepline{0.5pt}

\usepackage{ucs}
\usepackage[utf8x]{inputenc}
\usepackage[T1]{fontenc}


\begin{document}
\def\arraystretch{1.5}
\begin{titlepage}
\begin{center}
\vspace{10em}
\includegraphics[scale=2]{jbomberman}
\vspace{10em}
\end{center}
\begin{center}
\huge {Projekt: JBomberman} \\
\huge {Systemtests}
\end{center}
\begin{center}
\vspace{10em}
\LARGE {Pascal Kistler} \\
\LARGE {Silvan Adrian} \\
\LARGE {Fabian Binna}
\end{center}

\end{titlepage}

\newpage
\section{Änderungshistorie}
\label{sec:Änderungen}

\begin{tabularx}{\linewidth}{l l l l}
\textbf{Datum} & \textbf{Version} & \textbf{Änderung}  & \textbf{Autor} \\
\hline
\textbf{09.03.15} & 1.00 & Erstellung des Dokuments & Gruppe \\
\textbf{20.04.15} & 1.01 & Struktur des Dokuments + Vorgaben & Silvan Adrian \\
\textbf{11.05.15} & 1.02 & Testprotokoll RC1 & Silvan Adrian \\
\textbf{24.05.15} & 1.03 & Testprotokoll RC2 & Silvan Adrian \\
\textbf{27.05.15} & 1.04 & Vorbereitung Abgabe & Silvan Adrian\\
\end{tabularx}

\newpage
\tableofcontents
\newpage
\section{Einführung}
\subsection{Zweck}
Dieses Dokument beinhaltet alle Informationen zu den Systemtest von JBomberman.
Hier soll festgehalten werden, was getestet wird und was als Ergebnis herauskommt.
\subsection{Gültigkeitsbereich}
Dieses Dokument ist während des ganzen Projekts gültig und wird laufend aktualisiert.
\subsection{Definitionen und Abkürzungen}
Siehe Glossar
\subsection{Referenzen}
Glossar
Projektplan
\subsection{Übersicht}
Es sollen Tests für verschiedene Releases erstellt werden, dabei werden für 
jedes Release Systemspezifikationen erstellt und das dazugehörige Testprotokoll 
eingefügt.
\section{Systemtests RC1}
\subsection{Übersicht}
In dieser Version werden die Grundfunktionen getestet, welche schlussendlich auch 
für den Usability Test RC1 gebraucht werden
\subsection{Voraussetzungen}
Um JBomberman spielen zu können braucht man mindestens einen Mitspieler
\subsection{Vorbereitungen}
Server mit RabbitMQ Broker und JBomberman Server muss gestartet sein und 
erreichbar. 
(Port: 5672 und 15672 müssen offen sein).
\subsection{UC01: JBomberman spielen}
\subsubsection{Systemtests}
\begin{tabularx}{\linewidth}{l X X}
  \bf{Test} & \bf{Beschreibung} & \bf{Soll Verhalten} \\
  \bf{T1} & Bewegung in alle Himmelsrichtungen möglich & 
  Bomberman lässt sich in alle Himmelsrichtungen steuern\\
  \bf{T2} & Kollisionen zwischen Bomberman und SolidBlocks/DestroyableBlocks 
  findet statt & Bomberman kann die Blocks nicht durchqueren \\
  \bf{T3} & Bomberman legt Bombe & Bombe wird gelegt \\
  \bf{T4} & gelegt Bome explodiert nach (momentan noch 2s) & Bombe explodiert \\
  \bf{T5} & Feuer der Explosion verteilen sich (wenn SolidBlock im Weg nicht) & Feuer der 
  Explosion verteilt sich.\\
  \bf{T6} & Destroyable Block wird durch Explosion zerstört & Destroyable Block wird 
  zerstört \\
  \bf{T7} & Bomberman wird durch Explosion aus dem Spiel geworfen & 
  Bomberman wird aus dem Spiel geworfen\\
  \bf{T8} & letzter Bomberman gewinnt (Spiel wird geschlossen) & Spiel wird 
  geschlossen \\
  \bf{T8} & Wenn Runde abgeschlossen werden Spieler in die ConnectionPanel zurückgeführt & 
  Spieler werden zum ConnectionPanel zurückgeführt \\
  \bf{T110} & Auf dem Server kann ein neues Spiel gestartet werden & Neues Spiel 
  kann gestartet werden \\
  \bf{T11} & Wenn 2 Spieler bereit wird Timer gestartet bis Spiel gestartet 
  wird & Timer wird gestartet \\
  \bf{T12} & Spieler kann sich umentscheiden (Timer wird zurückgesetzt) & Timer 
  fängt neu an \\
  \bf{T13} & Spieler, die in der Lobby nicht bereit waren werden entfernt & 
  Spieler werden entfernt \\
  \bf{T14} & Wenn Spiel gestartet kann kein Spieler mehr sich auf den Server 
  verbinden & Verbinden nicht möglich \\
  \bf{T15} & Wenn Lobby voll kann der Spieler nicht connecten & Spieler kann 
  nicht connecten und Meldung Server Full kommt zurück \\
  \bf{T16} & PowerUps kommen zutage & PowerUp erscheinen \\
  \bf{T17} & PowerUp Funktion funktioniert & PowerUp funktioniert \\
\end{tabularx}

\subsection{Testprotokoll}
\begin{tabularx}{\linewidth}{l l X}
  \bf{Test} & \bf{Ist} & \bf{Beschreibung} \\
  \bf{T1} & OK & Bomberman bewegt sich in alle Richtungen\\
  \bf{T2} & OK &  Kollison findet statt\\
  \bf{T3} & OK  & Bombe wird gelegt (allerdings noch nicht sehr präzise)\\
  \bf{T4} & OK &  Bombe explodiert\\
  \bf{T5} & OK & Feuer verteilt sich nicht über Solid Blocks\\
  \bf{T6} & OK &  Block wird zerstört\\
  \bf{T7} & OK & Bomberman verschwindet\\
  \bf{T8} & OK & Spiel wird bebendet und es kann von neuem Connected werden\\
  \bf{T9} & OK & Werden zurückgeführt\\
  \bf{T10} & OK & Ist möglich\\
  \bf{T11} & OK& Spiel wird gestartet und nur Ready Spieler eingebunden\\
  \bf{T12} & OK &  Timer wird zurückgesetzt\\
  \bf{T13} & OK &  Spieler werden entfernt\\
  \bf{T14} & Failed & Kann zwar nicht connected werden gibt aber kein Rückmeldung \\
  \bf{T15} & OK & Meldung Server Full kommt zurück\\
  \bf{T16} & OK & PowerUps erscheinen\\
  \bf{T17} & OK & PowerUps funktionieren\\
\end{tabularx}
\subsection{Beschreibung}
Testcase 14 ist fehlgeschlagen, wird aber aus Zeitgründen nicht 
implementiert (da zwischen zu vielen cases unterschieden werden muss)
\section{Systemtests RC2}
\subsection{Übersicht}
In dieser Version werden die erweiterten Funktionen getestet (Matchmaking -> 
Spiel über mehrere Runden).
Die erweiterte Funktionen werden auch einem Usability Test unterzogen.
\subsection{Voraussetzungen}
Um JBomberman spielen zu können braucht man mindestens einen Mitspieler
\subsection{Vorbereitungen}
Server mit RabbitMQ Broker und JBomberman Server muss gestartet sein und 
erreichbar. 
(Port: 5672 und 15672 müssen offen sein).
\subsection{UC01: JBomberman spielen}
\subsubsection{Systemtests}
\begin{tabularx}{\linewidth}{l X X}
  \bf{Test} & \bf{Beschreibung} & \bf{Soll Verhalten} \\
 \bf{T1} & Matchmaking über mehrere Runden & Spiel kann über mehrere 
 Runden gespielt werden \\
 \bf{T2} & Banner für Rundengewinner wird angezeigt & Banner wird angezeigt \\
 \bf{T3} & Banner für Spielgewinner wird angezeigt & Banner wird angezeigt \\
 \bf{T4} & Banner wenn für kein Gewinner in der Spielrunde & Banner wird 
 angezeigt\\
 \bf{T5} & Spieler Score zählt nach jedem Gewinn eins Hoch & Score 1 höher als 
 zuvor \\
 \bf{T6} & Timer läuft aus PowerUp's werden auf höchste Stufe gesetzt & 
 PowerUp's auf höchster Stufe \\
 \bf{T7} & Nach Spielschluss wird das Connection Panel geöffnet & Connection 
 Panel wird angezeigt\\
 \bf{T8} & Es können mehrere Spiele hintereinander gespielt werden &
 Mehrere Spiele können gespielt werden \\
 \bf{T9} & Bomben werden durch Explosionen anderer Bomben zerstört &
 Bomben werden zerstört \\
 
 \end{tabularx}

\subsection{Testprotokoll}
\begin{tabularx}{\linewidth}{l l X}
  \bf{Test} & \bf{Ist} & \bf{Beschreibung} \\
  \bf{T1} & OK & Spiel kann über mehrere Runden gespielt werden\\
  \bf{T2} & OK &  Banner wird angezeigt \\
  \bf{T3} & OK  & Banner wird angezeigt \\
  \bf{T4} & OK &  Banner wird angezeigt \\
  \bf{T5} & OK & Score 1 höher als zuvor\\
  \bf{T6} & OK &   PowerUp's auf höchster Stufe \\
  \bf{T7} & OK & Connection Panel wird angezeigt\\
  \bf{T8} & OK & Mehrere Spiele können gespielt werden\\
  \bf{T9} & OK & Bomben werden zerstört\\
\end{tabularx}
\subsection{Beschreibung}
Testcases haben alle geklappt und werden noch genauer getestet bei dem Usability 
Test.
\end{document}