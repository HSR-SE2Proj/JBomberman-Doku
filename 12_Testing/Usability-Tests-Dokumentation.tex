\documentclass[11pt]{scrartcl}

\title{Usability Tests: JBomberman}
\author{Silvan Adrian \\ Fabian Binna \\ Pascal Kistler}
\date{\today{}}

\usepackage[ngerman]{babel}
\usepackage[automark]{scrpage2}
\usepackage{hyperref}
\usepackage{color}
\usepackage[normalem]{ulem}
\usepackage{scrpage2}
\usepackage{graphicx,xcolor}
\usepackage{tabularx}
\graphicspath{ {./images/} }
\pagestyle{scrheadings}

\clearscrheadfoot
\ihead{\includegraphics[scale=0.4]{jbomberman}}
\ohead{Projekt: JBomberman}
\ifoot{Pojektantrag: JBomberman}
\cfoot{Version: 1.02}
\ofoot{Datum: 27.04.15}
\setheadsepline{0.5pt}
\setfootsepline{0.5pt}

\usepackage{ucs}
\usepackage[utf8]{inputenc}
\usepackage[T1]{fontenc}


\begin{document}
\def\arraystretch{1.5}
\begin{titlepage}
\begin{center}
\vspace{10em}
\includegraphics[scale=2]{jbomberman}
\vspace{10em}
\end{center}
\begin{center}
\huge {Projekt: JBomberman} \\
\huge {Usability Tests}
\end{center}
\begin{center}
\vspace{10em}
\LARGE {Pascal Kistler} \\
\LARGE {Silvan Adrian} \\
\LARGE {Fabian Binna}
\end{center}

\end{titlepage}

\newpage
\section{Änderungshistorie}
\label{sec:Änderungen}

\begin{tabularx}{\linewidth}{l l l l}
\textbf{Datum} & \textbf{Version} & \textbf{Änderung}  & \textbf{Autor} \\
\hline
\textbf{09.03.15} & 1.00 & Erstellung des Dokuments & Gruppe \\
\textbf{20.04.15} & 1.01 & Dokumentstruktur & Silvan Adrian \\
\textbf{27.04.14} & 1.02 & Aufbau von Fragebogen eingefügt (Formular) & Silvan 
Adrian \\
\end{tabularx}

\newpage
\tableofcontents
\newpage
\section{Einführung}
\subsection{Zweck}
Dieses Dokument beinhaltet die Spezifikationen für die Usability Tests.
\subsection{Gültigkeitsbereich}
Dieses Dokument ist während des ganzen Projekts gültig und wird laufend aktualisiert.
\subsection{Definitionen und Abkürzungen}
Siehe Glossar
\subsection{Referenzen}
Glossar
Projektplan
\subsection{Übersicht}
Es sollen mit verschiedenen Personen Usability Tests durchgeführt werden.

\section{Einführung}
Es sollen mehrere Personen an einem Usability Test teilnehmen und über 
Fragebögen sollen Daten gesammelt werden, wie gut das Spiel ankommt.

\subsection{Vorgaben}
Die Usability Tests werden immer in der Anwesenheit eines Entwicklers 
vorgenommen und orientieren sich an den 5 E's von
Effective
Efficient
Engaging
Error tolerant
Easy to learn -> dazu Fragebogen erstellen und auf die Unterthemen eingehen.

+ Externes Design -> wie wirkt das Externe Design
+ 

\subsection{Usability Test Vorgang}
Da bei JBomberman für die Tests mehr als 2 Personen von Vorteil sind, werden die 
Tests je nach Anzahl der Teilnehmer in einer grösseren Gruppe veranstaltet.
\begin{itemize}
  \item Test wird mit Entwickler vor Ort durchgeführt
  \item Feedback wird durch Tester ausgefüllt
  \item Kurzes Feedback Gespräch mit dem Entwickler vor Ort
\end{itemize}

\subsection{Test-Personen}
\begin{tabularx}{\linewidth}{l X}
  \bf{Person} & \bf{Datum} \\
  
\end{tabularx}


\subsection{Fragebogen}
\begin{Form}
\subsubsection{Allgemein}
\begin{tabularx}{\linewidth}{l l}
  \bf{Name} & \bf{Datum}\\
 \TextField[name=Name,width=5cm,  bordercolor={red}, borderstyle=U, 
 value={}, backgroundcolor={0.95 0.95 0.95}]{} &
 \TextField[name=Datum,width=5cm,  bordercolor={red}, borderstyle=U, 
 value={}, backgroundcolor={0.95 0.95 0.95}]{}
\end{tabularx}
\subsubsection{Effektivität}
\subsubsection{Effizienz}
\subsubsection{Sympathie}
\subsubsection{Fehlertoleranz}
\subsubsection{Erlernbarkeit}
\subsubsection{Anmerkungen}
  \TextField[multiline,width=\textwidth,  height=3.25cm,borderstyle=D, 
  bordercolor={red}, value={}, backgroundcolor={0.95 0.95 0.95}]{}

\end{Form}
\end{document}