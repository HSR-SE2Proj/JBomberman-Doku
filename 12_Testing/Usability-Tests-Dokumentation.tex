\documentclass[11pt]{scrartcl}

\title{Usability Tests: JBomberman}
\author{Silvan Adrian \\ Fabian Binna \\ Pascal Kistler}
\date{\today{}}

\usepackage[ngerman]{babel}
\usepackage[automark]{scrpage2}
\usepackage{hyperref}
\usepackage{color}
\usepackage[normalem]{ulem}
\usepackage{scrpage2}
\usepackage{graphicx,xcolor}
\usepackage{tabularx}
\graphicspath{ {./images/} }
\pagestyle{scrheadings}

\clearscrheadfoot
\ihead{\includegraphics[scale=0.4]{jbomberman}}
\ohead{Projekt: JBomberman}
\ifoot{Usability Tests: JBomberman}
\cfoot{Version: 1.04}
\ofoot{Datum: 11.05.15}
\setheadsepline{0.5pt}
\setfootsepline{0.5pt}

\usepackage{ucs}
\usepackage[utf8]{inputenc}
\usepackage[T1]{fontenc}


\begin{document}
\def\arraystretch{1.5}
\begin{titlepage}
\begin{center}
\vspace{10em}
\includegraphics[scale=2]{jbomberman}
\vspace{10em}
\end{center}
\begin{center}
\huge {Projekt: JBomberman} \\
\huge {Usability Tests}
\end{center}
\begin{center}
\vspace{10em}
\LARGE {Pascal Kistler} \\
\LARGE {Silvan Adrian} \\
\LARGE {Fabian Binna}
\end{center}

\end{titlepage}

\newpage
\section{Änderungshistorie}
\label{sec:Änderungen}

\begin{tabularx}{\linewidth}{l l l l}
\textbf{Datum} & \textbf{Version} & \textbf{Änderung}  & \textbf{Autor} \\
\hline
\textbf{09.03.15} & 1.00 & Erstellung des Dokuments & Gruppe \\
\textbf{20.04.15} & 1.01 & Dokumentstruktur & Silvan Adrian \\
\textbf{27.04.15} & 1.02 & Aufbau von Fragebogen eingefügt (Formular) & Silvan Adrian \\
\bf{03.05.15} & 1.03 & Erste Einträge für den Fragebogen & Silvan Adrian \\
\bf{11.05.15} & 1.04 & Einträge erweitert und verbessert & Silvan Adrian \\
\end{tabularx}

\newpage
\tableofcontents
\newpage
\section{Einführung}
\subsection{Zweck}
Dieses Dokument beinhaltet die Spezifikationen für die Usability Tests.
\subsection{Gültigkeitsbereich}
Dieses Dokument ist während des ganzen Projekts gültig und wird laufend aktualisiert.
\subsection{Definitionen und Abkürzungen}
Siehe Glossar
\subsection{Referenzen}
\begin{itemize}
  \item Glossar
  \item Projektplan
  \item Quesenberry
\end{itemize}
\subsection{Übersicht}
Es sollen mit verschiedenen Personen Usability Tests durchgeführt werden.

\section{Einführung}
Es sollen mehrere Personen an einem Usability Test teilnehmen und über 
Fragebögen sollen Daten gesammelt werden, wie gut das Spiel ankommt.

\subsection{Vorgaben}
Die Usability Tests werden immer in der Anwesenheit eines Entwicklers 
vorgenommen und orientieren sich an den 5 E's von Quesenberry.
\begin{itemize}
  \item \textbf{Effective} - How completely and accurately the work or experience 
  is completed or goals reached
  \item \textbf{Efficient} - How quickly this work can be completed
  \item \textbf{Engaging} - How well the interface draws the user into the 
  interaction and how pleasant and satisfying it is to use
  \item \textbf{Error tolerant} - How well the product prevents errors and can help the 
  user recover from mistakes that do occur
  \item \textbf{Easy to learn} - How well the product supports both the initial orientation and continued 
  learning throughout the complete lifetime of use
\end{itemize}
\subsection{Usability Test Vorgang}
Da bei JBomberman für die Tests mehr als 2 Personen von Vorteil sind, werden die 
Tests je nach Anzahl der Teilnehmer in einer grösseren Gruppe veranstaltet.
\begin{itemize}
  \item Test wird mit Entwickler/Entwicklern vor Ort durchgeführt
  \item Feedbackfragebogen wird durch Tester ausgefüllt
  \item Kurzes Feedback Gespräch mit dem Entwickler vor Ort
\end{itemize}

\subsection{Test-Personen}
\begin{tabularx}{\linewidth}{l X}
  \bf{Person} & \bf{Datum} \\
  Luis & 11.05.15\\
\end{tabularx}
\newpage

\subsection{Fragebogen}
\begin{Form}
\begin{tabularx}{\linewidth}{l l l l}

 \bf{Name} &\TextField[name=Name,width=5cm,  bordercolor={red}, borderstyle=U, 
 value={}, backgroundcolor={0.95 0.95 0.95}]{} &
    \bf{Datum} &
 \TextField[name=Datum,width=5cm,  bordercolor={red}, borderstyle=U, 
 value={}, backgroundcolor={0.95 0.95 0.95}]{} \\
\end{tabularx}
\begin{tabularx}{\linewidth}{l | X | l | X}
 \multicolumn{4}{l}{ \bf{Bewertungen}}\\
 \hline
  1 & Trifft gar nicht zu & 4 & Trifft grösstenteils zu \\
  \hline
  2 & Trifft ein bisschen zu & 5 & Trifft in hohem masse zu \\
  \hline
  3 & Trifft mehr oder weniger zu & N/A & nicht anwendbar/beurteilbar \\
  \hline
\end{tabularx}
\newline
\begin{tabularx}{\linewidth}{l | p{8cm}| X | X | X  | X | X | l}
 \bf{A} & \bf{Effektivität} & \bf{1} & \bf{2} & \bf{3} & \bf{4} & \bf{5} & \bf{N/A}\\
  \hline
  A1 & Ich kann das Spiel spielen & & & & & \\
  \hline
  A2 & Ich kann die Spielfigur steuern & & & & & & \\
  \hline
\end{tabularx}
\newline
\begin{tabularx}{\linewidth}{l | p{8cm}| X | X | X  | X | X | l}
 \bf{B} & \bf{Effizienz} & \bf{1} & \bf{2} & \bf{3} & \bf{4} & \bf{5} & \bf{N/A}\\
  \hline
  B1 & Ich kann schnell ein Spiel starten & & & & & \\
  \hline
  B2 & Ich kann das Spiel effizient bedienen & & & & & & \\
  \hline
\end{tabularx}
\newline
\begin{tabularx}{\linewidth}{l | p{8cm}| X | X | X  | X | X | l}
 \bf{C} & \bf{Zufriedenheit} & \bf{1} & \bf{2} & \bf{3} & \bf{4} & \bf{5} & \bf{N/A}\\
  \hline
  C1 & Ich hatte Spass beim spielen & & & & & \\
  \hline
  C2 & Ich bin grundsätzlich zufrieden mit der Applikation& & & & & & \\
  \hline
\end{tabularx}
\newline
\begin{tabularx}{\linewidth}{l | p{8cm}| X | X | X  | X | X | l}
 \bf{E} & \bf{Fehlertoleranz} & \bf{1} & \bf{2} & \bf{3} & \bf{4} & \bf{5} & \bf{N/A}\\
  \hline
  E1 & Fehler beeinträchtigen nicht das Spiel & & & & & \\
  \hline
  E2 & Eine falsche Eingabe hat nicht einen Absturz zur Folge & & & & & & \\
  \hline
  E3 & Fehlertoleranz ist grundsätzlich vorhanden & & & & &  &\\
  \hline
\end{tabularx}
\newline
\begin{tabularx}{\linewidth}{l | p{8cm}| X | X | X  | X | X | l}
 \bf{D} & \bf{Erlernbarkeit} & \bf{1} & \bf{2} & \bf{3} & \bf{4} & \bf{5} & \bf{N/A}\\
  \hline
  D1 & Das Spiel ist einfach erlernbar & & & & & \\
  \hline
  D2 & Ich kann das Spiel einfach bedienen & & & & & & \\
  \hline
\end{tabularx}
\newline \newline
\bf{Anmerkungen} \newline
  \TextField[multiline,width=\textwidth,  height=2.5cm,borderstyle=D, 
  bordercolor={red}, value={}, backgroundcolor={0.95 0.95 0.95}]{}

\end{Form}
\end{document}