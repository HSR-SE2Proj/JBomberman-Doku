\documentclass[11pt]{scrartcl}

\title{Projektplan: JBomberman}
\author{Silvan Adrian \\ Fabian Binna \\ Pascal Kistler}
\date{\today{}}

\usepackage[ngerman]{babel}
\usepackage[automark]{scrpage2}
\usepackage{hyperref}
\usepackage{color}
\usepackage[normalem]{ulem}
\usepackage{scrpage2}
\usepackage{graphicx}
\usepackage{tabularx}
\usepackage{longtable,tabu}
\graphicspath{ {./images/} }
\pagestyle{scrheadings}

\clearscrheadfoot
\ihead{\includegraphics[scale=0.4]{jbomberman}}
\ohead{Projekt: JBomberman}
\ifoot{Iterationsplan: JBomberman}
\cfoot{Version: 1.02}
\ofoot{Datum: 06.05.15}
\setheadsepline{0.5pt}
\setfootsepline{0.5pt}

\usepackage{ucs}
\usepackage[utf8]{inputenc}
\usepackage[T1]{fontenc}


\begin{document}
\def\arraystretch{1.5}
\begin{titlepage}
\begin{center}
\vspace{10em}
\includegraphics[scale=2]{jbomberman}
\vspace{10em}
\end{center}
\begin{center}
\huge {Projekt: JBomberman} \\
\huge {Iterationsplan}
\end{center}
\begin{center}
\vspace{10em}
\LARGE {Pascal Kistler} \\
\LARGE {Silvan Adrian} \\
\LARGE {Fabian Binna}
\end{center}

\end{titlepage}

\newpage
\section{Änderungshistorie}
\label{sec:Änderungen}

\begin{tabularx}{\linewidth}{l l X l}
\textbf{Datum} & \textbf{Version} & \textbf{Änderung}  & \textbf{Autor} \\
\hline
\textbf{09.03.15} & 1.00 & Erstellung des Dokuments & Gruppe \\
\bf{20.04.15} & 1.01 & Einfügen von Arbeitspaketen zu einzelnen Iterationen & 
Silvan Adrian\\
\bf{06.05.15} & 1.02 & C1 hinzugefügt + generelles am Dokument & Silvan Adrian\\
\end{tabularx}

\newpage
\tableofcontents
\newpage
\section{Einführung}
\subsection{Zweck}
Dieses Dokument soll einen Überblick über alle Iterationen geben und eine 
Referenz zu bereits erledigten Iterationen geben und für geplante.
Dabei werden alle geplanten Arbeitspakete aufgelistet und der geschätzte Aufwand 
dazu + der wirklich gebrauchte Aufwand Schlussendlich.
Übergeordnete Tickets werden nicht aufgelistet, da diese einen Arbeitsaufwand 
von 0 besitzen.
\subsection{Referenzen}
\begin{itemize}
  \item Projektplan
  \item Redmine
\end{itemize}
\section{Übersicht}
\subsection{Phasenübersicht}
Es bestehen 8 Phasen/Iterationen.
\newline
\begin{tabularx}{\linewidth}{X X}
  \bf{Name} & \bf{Datum} \\
  \hline
  Inception & 20.02.2015 \\
  Elaboration 1 & 08.03.2015 \\
  Elaboration 2 & 22.03.2015 \\
  Elaboration 3 & 05.04.2015 \\
  Construction 1 & 19.04.2015 \\
  Construction 2 & 10.05.2015 \\
  Construction 3 & 24.05.2015 \\
  Transition 	& 31.05.2015 \\
\end{tabularx}
\subsection{Planungübersicht}

\section{Elaboration 1 (23.02.15 - 08.03.15)}
\subsection{Arbeitsergebnisse}
Anforderungen definiert Technologien ausgewählt und Projektplan 
fertig gestellt. \newline Erster Prototyp 
zum Testen mit Grundfunktionen (Befehle an Server senden).
\subsection{Arbeitspakete Soll}
  \begin{tabularx} {\linewidth}{lXl}
    \bf{Ref.}  & \bf{Beschreibung} & \bf{Soll[h]} \\
    \hline
    30 & Entwicklungsumgebung aufsetzen & 10 \\
    40 & Projektplan & 15 \\
    43 & Netzwerkprototyp & 6 \\
    44 & DomainModel und Zustandsdiagramm skizzieren & 3 \\
    46 & Dokumente in LaTex & 8 \\
    51 & Erste GUI Skizzen & 8 \\
    \hline
    & & 50
  \end{tabularx}
  
  \subsection{Arbeitspakete Ist}
  \begin{tabularx} {\linewidth}{lXll}
    \bf{Ref.}  & \bf{Beschreibung} & \bf{Soll[h]}  & \bf{Ist[h]}\\
    \hline
    30 & Entwicklungsumgebung aufsetzen & 10 & 16 \\
    40 & Projektplan & 15 & 19\\
    43 & Netzwerkprototyp & 6 & 11.50\\
    44 & DomainModel und Zustandsdiagramm skizzieren & 3 & 3.5 \\
    46 & Dokumente in LaTex & 8 & 8\\
    51 & Erste GUI Skizzen & 8 & 8\\
    \hline
    & & 50 & 66
    \end{tabularx}
    
    
    \subsection{Diagramm}
    folgt
\section{Elaboration 2 (09.03.15 - 22.03.15)}
\subsection{Zeitraum}
\subsection{Arbeitsergebnisse}
Use Cases aufschreiben, dazu das Spielprinzip genauer beschreiben. 
Nicht Funktionale Anforderungen bestimmen.
\subsection{Arbeitspakete Soll}
  \begin{tabularx} {\linewidth}{lXl}
    \bf{Ref.}  & \bf{Beschreibung} & \bf{Soll[h]} \\
    \hline
    %41 & Anforderungsanalyse & 18.50 \\
    47 & Projektplan verbessern & 1.5 \\
    48 & Grafiken & 8 \\
    49 &  Backup & 3 \\
    50 & Spielprinzip detailliert & 5 \\
    52 &  GUI Use Cases & 8 \\
    53 & Domain Modell Spielobjekte & 4\\
    54 & Domain Model Anforderungsanalyse & 3\\
    55 & Nicht funktionale Anforderungen & 2.5 \\
    56 & Konzeptbeschreibung & 2 \\
    57 & Risikoanalyse: Game Architektur  & 6 \\
    58 & SSD USe Case JBomberman spielen & 4 \\
    59 & Contract Use Case JBomberman spielen & 4\\
    60 & 	Domain Analyse Dokumentation & 3 \\
    \hline
    & & 54
  \end{tabularx}
  
  \subsection{Arbeitspakete Ist}
  \begin{tabularx} {\linewidth}{lXll}
    \bf{Ref.}  & \bf{Beschreibung} & \bf{Soll[h]}  & \bf{Ist[h]}\\
    \hline
   %41 & Anforderungsanalyse & 18.50 & 0\\
    47 & Projektplan verbessern & 1.5 & 1\\
    48 & Grafiken & 8 & 7\\
    49 &  Backup & 3 & 3\\
    50 & Spielprinzip detailliert & 5 & 5\\
    52 &  GUI Use Cases & 8 & 8\\
    53 & Domain Modell Spielobjekte & 4 & 7\\
    54 & Domain Model Anforderungsanalyse & 3 & 5\\
    55 & Nicht funktionale Anforderungen & 2.5 & 1.5\\
    56 & Konzeptbeschreibung & 2 & 1.5\\
    57 & Risikoanalyse: Game Architektur  & 6 & 2\\
    58 & SSD USe Case JBomberman spielen & 4 & 4\\
    59 & Contract Use Case JBomberman spielen & 4 & 4\\
    60 & 	Domain Analyse Dokumentation & 3 & 3 \\
    \hline
    & & 54 & 52
    \end{tabularx}

\subsection{Diagramm}
folgt
\section{Elaboration 3 (23.03.15 - 05.04.15)}
\subsection{Arbeitsergebnisse}
Architektur Prototyp (Kommunikationsverfahren zwischen Client und Server) 
fertiggestellt. 
Performance Tests zum ausgewählten Verfahren zwischen Client und Server.
\subsection{Arbeitspakete Soll}
  \begin{tabularx} {\linewidth}{lXl}
    \bf{Ref.}  & \bf{Beschreibung} & \bf{Soll[h]} \\
    \hline
    %31 & 	Architekturprototyp & 20 \\
    64 & Notifications über Slack einrichten + refs und Zeit aufschreibung über Commits & 2 \\
    66 & Besprechung Architektur JBomberman + offene Fragen klären & 4 \\
    67 &  Grundgerüst Prototyp & 8 \\
    68 & Architekturprototyp Grundgerüst Packagediagramm & 4 \\
    69 &  Architekturprototyp Refactoring der Domain & 4 \\
    70 & GUI Prototyp & 8\\
    71 & 	Verbesserungen anhand von Reviewprotokoll & 1.5\\
    72 & Rendering mit Java2D und BufferStrategy & 2 \\
    73 & Architektur Dokument & 14 \\
    74 & Glossar erstellen  & 1.5 \\
    76 & State Diagrams für Client und Server & 3 \\
    77 & Deployment Diagramm & 1\\
    78 & 	Prozesse und Threads & 1 \\
    79 & 	Packages Beschreibungen & 3 \\
    80 & Ant Builds & 8 \\
    81 & SSD GameLoop Client & 1 \\
    82 & Netzwerkübertragung & 1 \\
    83 & Package Diagramme erstellen & 4 \\
    85 & 	Findbugs & 2 \\
    86 & Ant Build Client/Server & 5 \\
    126 & JBomberman Projekt aufsetzen & 4 \\
    127 & Verbesserungen gemäss Review vom 9.04 & 2 \\
    128 & Tests für den Prototyp schreiben & 2 \\
    130 & Qualitätsmassnahmen	& 2 \\
    149 & Allgemeines & 3 \\
    \hline
    & & 77
  \end{tabularx}
  
  \subsection{Arbeitspakete Ist}
  \begin{tabularx} {\linewidth}{lXll}
    \bf{Ref.}  & \bf{Beschreibung} & \bf{Soll[h]}  & \bf{Ist[h]}\\
    \hline
  %31 & 	Architekturprototyp & 20 & 0\\
    64 & Notifications über Slack einrichten + refs und Zeit aufschreibung über Commits & 2 & 4 \\
    66 & Besprechung Architektur JBomberman + offene Fragen klären & 4 & 5.75\\
    67 &  Grundgerüst Prototyp & 8 & 8\\
    68 & Architekturprototyp Grundgerüst Packagediagramm & 4 & 4\\
    69 &  Architekturprototyp Refactoring der Domain & 4 & 4\\
    70 & GUI Prototyp & 8 & 9\\
    71 & 	Verbesserungen anhand von Reviewprotokoll & 1.5 & 1.5\\
    72 & Rendering mit Java2D und BufferStrategy & 2 & 2\\
    73 & Architektur Dokument & 14 & 2.75\\
    74 & Glossar erstellen  & 1.5 & 1.25\\
    76 & State Diagrams für Client und Server & 3 & 2\\
    77 & Deployment Diagramm & 1 & 1\\
    78 & 	Prozesse und Threads & 1 & 1\\
    79 & 	Packages Beschreibungen & 3 & 8\\
    80 & Ant Builds & 8 & 8.5\\
    81 & SSD GameLoop Client & 1 & 1\\
    82 & Netzwerkübertragung & 1 & 1\\
    83 & Package Diagramme erstellen & 4 & 5\\
    85 & 	Findbugs & 2 & 2.25\\
    86 & Ant Build Client/Server & 5 & 4.75\\
    126 & JBomberman Projekt aufsetzen & 4 & 4\\
    127 & Verbesserungen gemäss Review vom 9.04 & 2 & 2\\
    128 & Tests für den Prototyp schreiben & 2 & 1.5 \\
    130 & Qualitätsmassnahmen	& 2 & 2\\
    149 & Allgemeines & 3 & 3 \\
    \hline
    & & 77 & 89.25
    \end{tabularx}

\subsection{Diagramm}
folgt
\section{Construction 1 (06.04.15 - 19.04.15)}
\subsection{Arbeitsergebnisse}
Client und Server Kommunikation umgesetzt. 
Basic Character movements.
\subsection{Arbeitspakete Soll}
\begin{longtabu} to \textwidth {
    X[1,l]
    X[10,l]
    X[1,l]}
    \bf{Ref.}  & \bf{Beschreibung} & \bf{Soll[h]}\\
    \hline
    87 & Timer &	0.50\\
	88	& Player Implementation	 & 1.00\\
	89	& Party Implementation	& 1.00\\
	%90	& ActionDispatcher	& 0.25\\
	%91	& ActionQueue	& 0.50\\
	93	& ImageManager Implementation & 1.50\\
	99	& LobbyFrame & 8.00\\
	100 & ClientController & 4.00\\
	%101	& Keyboard	& 1.00\\
	102 & GameClient	& 0.50\\
	%103	& SpriteManager	 & 1.50\\
	104 & Sprite & 4.00\\
	107	 & ServerGame & 0.25\\
	108	 & GameObjectManager & 1.50\\
	131	 & SpriteManager Implementation & 1.00\\
	132 & SpriteManager Microtests & 0.50\\
	133	& GameObjectManager Implementation & 1.00\\
	134	 & GameObjectManager Microtests & 0.50\\
	135	 & Player Microtesting & 1.00\\
	136	 & Party Microtesting & 1.00\\
	137 & ImageManager Microtesting & 1.00\\
	139	 & ActionQueue Implementation + Microtests	& 0.50\\
	140	 & ActionDispatcher Implementation & 0.25\\
	141	 & GameClient Grundimplementation & 0.50\\
	143 & Server Game Implementation & 0.25\\
	147	 & Allgemeines & 6.00 \\
	153	 & Timer Implementation & 0.50\\
	154	 & Keyboard Implementation & 1.00\\
	159	 & Ticket Bereinigung & 0.50\\
	\hline
	& & 35.00
\end{longtabu}
\subsection{Arbeitspakete Ist}

\begin{longtabu} to \textwidth {
    X[1,l]
    X[10,l]
    X[1,l]
     X[1,l]}
    \bf{Ref.}  & \bf{Beschreibung} & \bf{Soll[h]} & \bf{Ist[h]}\\
    \hline
  87	 & Timer & 0.50 & 0.50\\
	88 & Player Implementation & 1.00 & 0.50\\
	89 & Party Implementation & 1.00 & 1.00\\
	%90 & ActionDispatcher & 0.25 & 0\\
	%91 & ActionQueue & 0.50 & 0\\
	93 & ImageManager Implementation & 1.50 & 2.00\\
	99 & LobbyFrame & 8.00 & 8.50\\
	100	 & ClientController & 4.00 & 5.00\\
	%101	 & Keyboard	 & 1.00	 & 0\\
	102	 & GameClient & 0.50 & 0.50\\
	%103	 & SpriteManager & 1.50 & 0\\
	104	 & Sprite & 4.00 & 5.50\\
	107	 & ServerGame & 0.25 & 0.50\\
	%108	 & GameObjectManager & 1.50 & 0\\
	131	 & SpriteManager Implementation & 1.00 & 1.00\\
	132	 & SpriteManager Microtests & 0.50 & 0.50\\
	133	 & GameObjectManager Implementation & 1.00 &	0.50\\
	134	 & GameObjectManager Microtests & 0.50	& 0.50\\
	135	 & Player Microtesting & 1.00 & 1.00\\
	136 & Party Microtesting	 & 1.00 & 0.50\\
	137	 & ImageManager Microtesting &	1.00 & 1.00\\
	139	 & ActionQueue Implementation + Microtests &	0.50 & 0.50\\
	140	 & ActionDispatcher Implementation & 0.25	 & 0.25\\
	141	 & GameClient Grundimplementation & 0.50	& 0.50\\
	143 & Server Game Implementation & 0.25 & 0.25\\
	147 & Allgemeines & 6.00 & 4.75\\
	153	 & Timer Implementation & 0.50 & 0.50\\
	154	 & Keyboard Implementation & 1.00 & 1.00\\
	159	 & Ticket Bereinigung & 0.50 & 0.50\\
	\hline
	& & 35.00 & 37.25
\end{longtabu}

\subsection{Diagramm}
folgt
\section{Construction 2 (20.04.15 - 10.05.15)}

\section{Construction 3 (11.05.15 - 24.05.15)}

\section{Transition (25.05.15 - 31.05.15)}



\end{document}