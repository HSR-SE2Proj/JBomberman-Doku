\documentclass[landscape,10pt]{scrartcl}

\title{Projektplan: JBomberman}
\author{Silvan Adrian \\ Fabian Binna \\ Pascal Kistler}
\date{\today{}}

\usepackage[ngerman]{babel}
\usepackage[automark]{scrpage2}
\usepackage[landscape,a4paper,margin=0.5cm,footskip=1cm]{geometry}
\usepackage{hyperref}
\usepackage{color}
\usepackage[normalem]{ulem}
\usepackage{scrpage2}
\usepackage{tabularx}
\pagestyle{scrheadings}

\clearscrheadfoot
\ifoot{TeschnischeRisken: JBomberman}
\cfoot{Version: 1.00}
\ofoot{Datum: \today{}}
\setfootsepline{0.5pt}

\usepackage{ucs}
\usepackage[utf8]{inputenc}
\usepackage[T1]{fontenc}

\begin{document}
 {\large{Risikomanagement}}
 
   \begin{tabular}{l l }
     Projekt: & JBomberman \\
     Author: & Fabian Binna + Silvan Adrian \\
     Gewichteter Schaden & 15.75
   \end{tabular}
  
 \begin{table}[h]
\begin{tabularx}{\textwidth}{l X X l l l X X}
\textbf{Nr} & \textbf{Titel} & \textbf{Beschreibung}  & 
\textbf{Schaden[h]} & \textbf{Wahrscheinlichkeit} & 
\textbf{gew. Schaden} & \textbf{Vorbeugung} &
 \textbf{Verhalten beim eintreten} \\
\hline
R1 & Grundgerüst (GameEngine) zu aufwändig & Implementierung 
wird zu  kompliziert 
und verlängert die Dauer des Projekts & 10 & 35\%
& 3.5 & Komplexität durch einfachere Konzepte verringern / Prototyping
Teamsitzung einberufen und neue, 
einfachere Konzepte besprechen
\end{tabularx}
\end{table}
\end{document}