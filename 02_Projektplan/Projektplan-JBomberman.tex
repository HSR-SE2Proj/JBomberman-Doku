\documentclass[11pt]{scrartcl}

\title{Projektplan: JBomberman}
\author{Silvan Adrian \\ Fabian Binna \\ Pascal Kistler}
\date{\today{}}

\usepackage[ngerman]{babel}
\usepackage[automark]{scrpage2}
\usepackage{hyperref}
\usepackage{color}
\usepackage[normalem]{ulem}
\usepackage{scrpage2}
\usepackage{graphicx}
\usepackage{tabularx}
\graphicspath{ {./images/} }
\pagestyle{scrheadings}

\clearscrheadfoot
\ihead{\includegraphics[scale=0.4]{jbomberman}}
\ohead{Projekt: JBomberman}
\ifoot{Pojektantrag: JBomberman}
\cfoot{Version: 1.07}
\ofoot{Datum: 07.03.15}
\setheadsepline{0.5pt}
\setfootsepline{0.5pt}

\usepackage{ucs}
\usepackage[utf8]{inputenc}
\usepackage[T1]{fontenc}


\begin{document}
\def\arraystretch{1.5}
\begin{titlepage}
\begin{center}
\vspace{10em}
\includegraphics[scale=2]{jbomberman}
\vspace{10em}
\end{center}
\begin{center}
\huge {Projekt: JBomberman} \\
\huge {Projektplan}
\end{center}
\begin{center}
\vspace{10em}
\LARGE {Pascal Kistler} \\
\LARGE {Silvan Adrian} \\
\LARGE {Fabian Binna}
\end{center}

\end{titlepage}

\newpage
\section{Änderungshistorie}
\label{sec:Änderungen}

\begin{tabularx}{\linewidth}{l l l l}
\textbf{Datum} & \textbf{Version} & \textbf{Änderung}  & \textbf{Autor} \\
\hline
\textbf{23.02.15} & 1.00 & Erstellung des Dokuments & Gruppe \\
\textbf{24.02.15} & 1.01 & Neue Texte eingefügt & Fabian Binna \\
\textbf{25.02.15} & 1.02 & Logo + Neue Texte & Silvan Adrian \\
\textbf{27.02.15} & 1.03 & Arbeitspakete & Fabian binna \\
\textbf{27.02.15} & 1.04 & Meilensteine + Phasen Iterationen & Silvan Adrian \\
\textbf{27.02.15} & 1.06 & Formatierungen + Design Anpassungen & Silvan Adrian \\
\textbf{27.02.15} & 1.06 & Korrekturen + Kostenvoranschlag & Fabian Binna \\
\textbf{07.03.15} & 1.07 & Migration auf LaTex und Verbesserungen & Silvan Adrian
\end{tabularx}

\newpage
\tableofcontents
\newpage

\section{Einführung}
\label{sec:Einführung}
\subsection{Zweck}
\label{sec:Zweck}
Dieses Dokument beschreibt die Planung des SE2 Projekts.
\subsection{Gültigkeitsbereich}
\label{sec:Gültigkeitsbereich}
Dieses Dokument ist während des ganzen Projekts gültig und wird laufend aktualisiert.
\subsection{Referenzen}
\label{sec:Refernzen}
Spielprinzip/Beschreibung: \href{http://de.wikipedia.org/wiki/Bomberman}{http://de.wikipedia.org/wiki/Bomberman} \\
Konzepte für Games: Spiele entwickeln für iPad, iPhone und iPod touch, Thomas Lucka, 2012, ISBN 978-3-446-43085-3

\section{Projektübersicht}
\label{sec:projektübersicht}
JBomberman ist ein Geschicklichkeitsspiel, das die gleichen Regeln wie das klassische Bomberman umfasst. Es gibt nur einen Multiplayer Modus, in dem bis zu vier Spieler gegeneinander antreten können. Der Multiplayer Modus ist über ein Netzwerk spielbar.

\subsection{Zweck und Ziel}
\label{sec:Zweck und Ziel}
Wir wollen gelerntes aus Prog1, Prog2, Uint1 und SE1 anwenden und erweiterte Programmierkonzepte erlernen. Hinzu kommen Module wie VSS und ParProg, die wir parallel besuchen und auch für die Durchführung des Projekts notwendig sind.
Unser Ziel ist es ein Software Projekt in einem Team erfolgreich durchzuführen.
 

\subsection{Primäre Features}
\label{sec:Primäre Features}
\begin{itemize}
    \item Spiellogik
    \item Netzwerkspiel
    \item Dedizierter Server
    \item Desktop Client
\end{itemize}

\subsection{Erweiterte Features}
\label{sec:Erweiterte Features}
\begin{itemize}
    \item Verschiedene Powerups
\end{itemize}

\subsection{Lieferumfang}
\label{sec:Lieferumfang}
\begin{itemize}
    \item Source-Code
    \item Dokumentation (Projekt und Software)
    \item Ausführbare Applikation
\end{itemize}

\subsection{Annahmen und Einschränkungen}
\label{sec:Annahmen und Einschränkungen}
Die Applikation wird lediglich für Desktops zur Verfügung gestellt, dabei sollen folgende Betriebssysteme (Mac OSX , Linux, Windows) unterstützt werden um eine möglichst grosse Userbasis anzusprechen.
Das Projekt soll nach Abschluss Open Source für Interessenten zur Verfügung gestellt werden.

\section{Projektorganisation}
\label{sec:Projektorganisation}
Jedes Teammitglied konzentriert sich auf seine eigenen Themengebiete, jedoch werden die Entscheidungen im Team besprochen und das Knowhow auf alle verteilt. Entscheidungen die das gesamte Projekt beeinflussen werden im Team besprochen und müssen von jedem akzeptiert werden.

\subsection{Organisationsstruktur}
\label{sec:Organisationsstruktur}
\textbf{Daniel Keller}
\begin{itemize}
    \item Projektbetreuer
\end{itemize}
\textbf{Silvan Adrian}
\begin{itemize}
    \item Projektleiter
    \item Serveradministrator
    \item Build Managment
    \item Entiwcklung
\end{itemize}
\textbf{Pascal Kistler}
\begin{itemize}
    \item Qulitätsmanagement
    \item Entwicklung
\end{itemize}
\textbf{Fabian Binna}
\begin{itemize}
    \item Software-Architekt
    \item Entwicklung
\end{itemize}

\subsection{Externe Schnittstellen}
\label{sec:Externe Schnittstellen}
Projektbetreuer: Daniel Keller (\href{mailto:d1keller@hsr.ch}{d1keller@hsr.ch})

\section{Managment Abläufe}
\label{sec:Managment Abläufe}
\subsection{Kostenvoranschlag}
\label{sec:Kostenvoranschlag}
Wir rechnen mit 420 Stunden Gesamtaufwand. 60 Stunden davon sind Reserve aufgrund von potenziellen Risiken.\\
\\
Gesamtaufwand: 420h \\
Davon Risikoaufwand: 60h \\
Aufwand pro Woche: 28h \\
Aufwand pro Woche pro Teammitglied: ~10h \\

\subsection{Phasen / Iterationen}
\label{sec:Phasen / Iterationen}

\begin{table}[h]
\begin{tabularx}{\textwidth}{l X l}
\textbf{Iteration} & \textbf{Beschreibung} & \textbf{Datum} \\
\hline
\textbf{Inception 1} & Projektantrag einreichen & 20.02.15 \\
\hline
\textbf{Elaboration 1} & Anforderungen definiert\, Technologien ausgewählt 
und Projektplan fertig gestellt. Erster Prototyp zum Testen mit Grundfunktionen (Befehle an Server senden). & 05.03.15 \\
\hline
\textbf{Elaboration 2} & Use Cases aufschreiben, dazu das Spielprinzip genauer beschreiben. 
Nicht Funktionale Anforderungen bestimmen. & 22.03.15 \\
\hline
\textbf{Elaboration 3} & Architektur Prototyp (Kommunikationsverfahren 
zwischen Client und Server), 
Performance Tests zum ausgewählten Verfahren. & 05.04.15 \\
\hline
\textbf{Construction 1} & Hauptfeatures implementieren
 (Kollisionsdetektion, Bomben Funktionieren, Powerups funktionieren) & 19.04.15 \\
\hline
\textbf{Construction 2} & Spielablauf und Matchmaking fertig. 
Falls genügend Zeit werden werden noch erweiterte Features eingebaut. & 03.05.15 \\
\hline
\textbf{Construction 3} & Suchablauf und Matchmaking fertig. Falls genügend Zeit werden
 noch erweiterte Features eingebaut. & 23.05.15 \\
\hline
\textbf{Transition} & Vorbereitung zur Schlusspräsentation 
und Abgabe. & 30.05.15 \\
\hline
\end{tabularx}
\end{table}

\subsection{Reviews}
\label{sec:Reviews}
\begin{table}[h]
\begin{tabularx}{\textwidth}{l X X}
\textbf{Datum} & \textbf{Reviews} & \textbf{Beschreibung} \\
\hline
\textbf{05.03.15} & Review Projektplan mit Zeitplan 
und aktuellen Iterationsplänen & MS1-RV Projektplan \\
\hline
\textbf{19.03.15} & Review der Anforderungsspezifikation
 und der Domainanalyse & MS2-RV Anforderungen 
 und Analyse \\
\hline
\textbf{09.04.15} & Zwischenpräsentation mit Demo eines
Architekturprototypen, Review & MS3-RV 
Ende Elaboration \\
\hline
\textbf{07.05.15} & Review von Architektur + Design
 und Architekturdokumenten & MS4-RV Architektur/Design \\
\hline
\textbf{28.05.15} & Präsentation und Demo der Software
& MS5-RV Schlusspräsentation und Schlussabgabe \\
\hline
\end{tabularx}
\end{table}

\subsection{Besprechungen}
Es werden regelmässige Besprechungen am 
Donnerstag/Montag morgen eingeplant um 
den Stand der jeweiligen arbeiten zu erfahren 
und gegebenenfalls Hilfestellung anzubieten 
,falls etwas nicht funktionieren sollte.

\section{Risikomanagement}
\label{Risikomanagement}
\subsection{Risiken}
\label{sec:Risiken}
Risiken werden im Dokument TechnischeRisiken-JBomberman.xslx beschrieben
\subsection{Umgang mit Risiken}
\label{sec:Umgang mit Risiken}
Für die Risiken werden Reserven eingeplant. Die Reserven werden direkt in die einzelnen Tickets eingerechnet. Falls Risiken eintreffen werden diese sofort kommuniziert. Jede Woche werden eingetroffene Risiken und potenzielle Risikogefahren diskutiert. Eingetroffene Risiken werden im Team besprochen und mögliche Lösungen evaluiert.

\section{Arbeitspakete}
\label{sec:Arbeitspakete}
Die Arbeitspakete werden in Redmine erstellt und gepflegt (siehe Screenshot). Lesender Zugriff ist anonym möglich, schreibender nur eingeloggt (Projekt ist öffentlich).
Link zum Redmine: \href{http://se2p.zonk.io/redmine}{http://se2p.zonk.io/redmine}

\section{Infrastruktur}
\label{sec:Infrastruktur}

\subsection{Entwicklungsinfrastruktur}
\label{sec:Entwicklungsinfrastruktur}
Aus persönlicher Präferenz ist die Infrastruktur breit gesät. Beim IDE konnten wir uns auf einen gemeinsamen Nenner bringen.

\begin{tabularx}{\textwidth}{c c c c}
\textbf{Name} & \textbf{Hardware} & \textbf{Betriebssystem} & \textbf{IDE} \\
\hline
Pascal Kistler & ASUS & Windows 8.1 & Eclipse Luna \\
\hline
Silvan Adrian & MacBook Pro & OSX 10.9.5 & Eclipse Luna \\
\hline
Fabian Binna & Lenovo T530s & Ubuntu 14.04.1 & Eclipse Luna \\
\hline
\end{tabularx}

\subsection{Tools/Software}
\label{sec:Tools/Software}
\begin{itemize}
    \item BuildServer: Jenkins
    \item CodeReview GitLab
    \item Konfigurationsmanagement GIT
\end{itemize}

\subsection{Kommunikationsmittel}
\label{sec:Kommunikationsmittel}
Ausserhalb des Reviewtermins wird mit folgenden Mitteln kommuniziert:
\begin{itemize}
    \item E-Mail
    \item Skype
    \item Redmine (Wiki Kommentare)
    \item GitLab (Kommentare)
    \item Whatsapp
\end{itemize}

\section{Qualitätsmassnahmen}
\label{sec:Qualitätsmassnahmen}
\subsection{Dokumentation}
\label{sec:Dokumentation}
Sämtliche Dokumente werden in Git gespeichert. So sind diese immer für alle verfügbar und können einfach versioniert und Änderungen nachvollzogen werden.
\subsection{Projektmanagement}
\label{sec:Projektmanagement}
Als Projektmanagementtool wird Redmine verwendet. Hier werden alle ausstehenden, laufenden und erledigen Tickets verwaltet.
\subsection{Entwicklung}
\label{sec:Entwicklung}
Der Source Code befindet sich ebenfalls in einem Git Repository und wird durch regelmässige Code Reviews überprüft.
\subsubsection{Vorgehen}
\label{sec:Vorgehen}
Jedem Commit wird eine Beschreibung hinzugefügt, danach knnen die Commits im Redmine dem passenden Ticket zugewiesen werden
\subsubsection{Unit Testing}
\label{sec:Unit Tesing}
Es sollten für jede Klasse / jedes Codemodul Junit Tests geschrieben und durchgeführt werden. Vor jedem push müssen alle Test erfolgreich durchlaufen werden.
\subsubsection{Code Reviews}
\label{sec:Code Reviews}
Durch Code Reviews wird sichergestellt, dass der Code den Style Guidelines entspricht.
Durch die Kontrolle durch ein oder zwei weitere Teammitglieder wird sichergestellt, dass die Codequalität stets auf hohem Niveau ist.
\subsubsection{Code Style Guidelines}
\label{sec:Code Style Guidelines}
Es werden die Code Style Guidelines aus den Modulen Prog1 und SE1 verwendet. Namen im Code (wie Variablen, Methoden, etc.) werden in Englisch benannt.
\subsection{Testen}
\label{sec:Testen}
\subsubsection{Komponententest}
\label{sec:Komponententest}
Zu jeder Komponente werden mehrere Test geschrieben. Diese stellen sicher, dass alle Komponenten ordnungsgemäss funktionieren und kein nicht funktionierender Code gepusht wird.
\subsection{Systemtest}
\label{sec:Systemtest}
Systemtest werden gegen Ende des Projekts erstellt. Sie garantieren, dass das gesamte System ordnungsgemäss funktioniert und alle Komponenten richtig zusammen arbeiten.


\end{document}