\documentclass[11pt]{scrartcl}

\title{Projektplan: JBomberman}
\author{Silvan Adrian \\ Fabian Binna \\ Pascal Kistler}
\date{\today{}}

\usepackage[ngerman]{babel}
\usepackage[automark]{scrpage2}
\usepackage{hyperref}
\usepackage{color}
\usepackage[normalem]{ulem}
\usepackage{scrpage2}
\usepackage{graphicx}
\usepackage{tabularx}
\graphicspath{ {./images/} }
\pagestyle{scrheadings}

\clearscrheadfoot
\ihead{\includegraphics[scale=0.4]{jbomberman}}
\ohead{Projekt: JBomberman}
\ifoot{DomainAnalyse: JBomberman}
\cfoot{Version: 1.01}
\ofoot{Datum: 20.03.15}
\setheadsepline{0.5pt}
\setfootsepline{0.5pt}

\usepackage{ucs}
\usepackage[utf8]{inputenc}
\usepackage[T1]{fontenc}


\begin{document}
\def\arraystretch{1.5}
\begin{titlepage}
\begin{center}
\vspace{10em}
\includegraphics[scale=2]{jbomberman}
\vspace{10em}
\end{center}
\begin{center}
\huge {Projekt: JBomberman} \\
\huge {DomainAnalyse}
\end{center}
\begin{center}
\vspace{10em}
\LARGE {Pascal Kistler} \\
\LARGE {Silvan Adrian} \\
\LARGE {Fabian Binna}
\end{center}

\end{titlepage}

\newpage
\section{Änderungshistorie}
\label{sec:Änderungen}

\begin{tabularx}{\linewidth}{l l l l}
\textbf{Datum} & \textbf{Version} & \textbf{Änderung}  & \textbf{Autor} \\
\hline
\textbf{09.03.15} & 1.00 & Erstellung des Dokuments & Gruppe \\
\textbf{20.03.15} & 1.01 & Vollendung des Dokuments & Fabian Binna \\
\end{tabularx}

\newpage
\tableofcontents
\newpage
\section{Einführung}
\label{sec:Einführung}

\subsection{Zweck}
\label{sec:Zweck}
Dieses Dokument beschreibt die Domainanalyse für das Projekt JBomberman.

\subsection{Gültigkeit}
\label{sec:Gültigkeit}
Dieses Dokument ist während des ganzen Projekts gültig und wird laufend aktualisiert.

\subsection{Übersicht}
\label{sec:Übersicht}
Dieses Dokument soll eine erste Analyse der Software zeigen. Das Strukturdiagramm stellt die wichtigsten Klassen dar, die miteinander interagieren müssen. Das Systemsequenzdiagramm beschreibt die wichtigsten Abläufe der Use Cases.

\newpage
\section{DomainModell}
\label{sec:DomainModell}
\subsection{Strukturdiagramm}
\label{sec:Strukturdiagramm}

\begin{center}
\includegraphics[scale=0.5]{Strukturdiagramm_JBomberman} 
\end{center}

\newpage

\subsection{Konzeptbeschreibung}
\label{sec:Konzeptbeschreibung}

\subsubsection{Game}
\label{sec:Game}
Die Klasse Game ist die oberste Klasse und kontrolliert den Ablauf des Spiels. \\\\
\textbf{Gamestate} \\
Der GameState weist auf den aktuelle Spielestatus hin. Je nachdem in welchem Status sich das Spiel befindet werden andere Routinen durchlaufen.

\subsubsection{Party}
\label{sec:party}
Die Party beinhaltet alle Spieler die aktuell am Spiel teilnehmen, und kontrolliert Zeit sowie die Spielrunden.
\\\\
\textbf{round}\\
Die aktuelle Spielrunde.
\\\\
\textbf{maxRounds}\\
Maximale Anzahl Runden die gespielt werden müssen.
\\\\
\textbf{time}\\
Time zählt die Zeit von 3 Minuten herunter. Bei Ende der Zeit kriegen alle Bombermans die vollen PowerUps.

\subsubsection{Player}
\label{sec:Player}
Der Player dient zur identifizierung der Teilnehmer. Jedem Spieler ist ein Bomberman zugewiesen. Die Messages werden über die jeweilige Playerinstanz dem korrekten Bomberman zugewiesen.
\\\\
\textbf{name}\\
Speichert den Namen des Spielers. Wird auch zur identifizierung verwendet.
\\\\
\textbf{score}\\
Speichert die aktuelle Punktezahl des Spielers.

\newpage

\subsubsection{Sprite}
\label{sec:Sprite}
Die Spriteklasse implementiert die Grundvoraussetzungen für jedes Spielelement. Jedes visuelle Element muss von der Spriteklasse erben.
\\\\
\textbf{Position}\\
Position ist eine Klasse, die x und y Koordinaten speichert.
\\\\
\textbf{image}\\
Image ist ein String, der den Namen des Bildes speichert, welcher intern zur identifizierung dient und bei einem ResourceManager die Referenz zum BufferedImage abholen kann.

\subsubsection{AnimatedSprite}
\label{sec:AnimatedSprite}
Das AnimatedSprite erbt von der Spriteklasse und implementiert weitere Funktionen, die ein animiertes Objekt ermöglicht.
\\\\
\textbf{animationSpeed}\\
Die Geschwindigkeit mit der durch die einzelnen Bildteile geschaltet wird.

\subsubsection{Bomberman}
\label{sec:Bomberman}
Die Bombermanklasse beschreibt den Zustand und die Fähigkeiten der Spielfigur.
\\\\
\textbf{movementSpeed }\\
Wie schnell sich ein Bomberman fortbewegen kann.
\\\\
\textbf{bombPower}\\
Wie weit reicht der Sprengradius.
\\\\
\textbf{bombs}\\ 
Anzahl Bomben die der Bomberman zeitgleich auf dem Spielfeld verteilen kann.
\\\\
\textbf{maxBombs}\\
Maximale Anzahl Bombem die der Bomberman tragen kann.

\subsubsection{PowerUp}
\label{sec:PowerUp}
Das PowerUp beschreibt die Art der Fähigkeit die den Bomberman verbessert.

\newpage

\subsubsection{Bomb}
\label{sec:Bomb}
Die Bombklasse repräsentiert die Bombe auf dem Spielfeld.
\\\\
\textbf{bombPower}\\
Die Reichweite der Explosionsarme.

\subsubsection{Explosion}
\label{sec:Explosion}
Beschreibt einen Teil eines Explosionsarms. Die bombPower der Bombe definiert wie viele Explosionen in jede Richtung erstellt werden müssen.

\subsubsection{DestroyableBlock}
\label{sec:DestroyableBlock}
Der DestroyableBlock kann von einer Bombe zerstört werden und hinterlässt manchmal ein PowerUp.


\section{Systemsequenzdiagramme}
\label{sec:Systemsequenzdiagramme}
\subsection{UC01: Bomberman spielen}
\label{sec:UC01: Bomberman spielen}
\begin{center}
\includegraphics[scale=0.3]{SystemSequenzDiagramm_JBomberman} 
\end{center}

\newpage
\section{Systemoperationen}
\label{sec:Systemoperationen}
\subsection{Client}
\begin{itemize}
\item connect(server: Address)
\item ready(player: Player)
\item pressed(key: Message)
\item disconnect()
\end{itemize}

\subsection{Server}
\begin{itemize}
    \item startGame()
    \item countdown(time: integer)
    \item update(mapChanges: Message)
    \item results(results: Array)
    \item endOfGame()
\end{itemize}


\subsection{Contracts}
\label{sec:Contracts}

\begin{tabularx}{\linewidth}{l X}
	\textbf{Operation} & connect(server : Address) \\
	\hline
	\textbf{Cross References} & UC01 \\
	\hline
	\textbf{Preconditions} & Eine Serveradressse wurde eingegeben \\
	\hline
	\textbf{Postconditions} & 
	\begin{minipage}{5in}
		\vskip 4pt
		\begin{itemize}
			\item Der Client ist mit dem Server verbunden
			\item Der Server kennt den Client
		\end{itemize}
		\vskip 4pt
	\end{minipage}  \\
\end{tabularx}
\\ \\
\begin{tabularx}{\linewidth}{l X}
	\textbf{Operation} & ready(player : Player) \\
	\hline
	\textbf{Cross References} & UC01 \\
	\hline
	\textbf{Preconditions} & Der Client befindet sich in der Lobby \\
	\hline
	\textbf{Postconditions} & 
	\begin{minipage}{4in}
		\vskip 4pt
		\begin{itemize}
			\item Der Status des Clients wurde auf dem Server auf \textbf{ready} gesetzt
		\end{itemize}
		\vskip 4pt
	\end{minipage}  \\
\end{tabularx}
\\ \\
\begin{tabularx}{\linewidth}{l X}
	\textbf{Operation} & startGame() \\
	\hline
	\textbf{Cross References} & UC01 \\
	\hline
	\textbf{Preconditions} & Alle Client sind ready \\
	\hline
	\textbf{Postconditions} & 
	\begin{minipage}{4in}
		\vskip 4pt
		\begin{itemize}
			\item Bei den Clients wurde die Spielumgebung gestartet
			\item Der Server befindet sich im Gamemode
		\end{itemize}
		\vskip 4pt
	\end{minipage}  \\
\end{tabularx}
\\ \\
\begin{tabularx}{\linewidth}{l X}
	\textbf{Operation} & countdown(time : integer) \\
	\hline
	\textbf{Cross References} & UC01 \\
	\hline
	\textbf{Preconditions} & Das Spiel wurde gestartet \\
	\hline
	\textbf{Postconditions} & 
	\begin{minipage}{4in}
		\vskip 4pt
		\begin{itemize}
			\item Ein Countdown wird den Spielern angezeigt
		\end{itemize}
		\vskip 4pt
	\end{minipage}  \\
\end{tabularx}
\\ \\
\begin{tabularx}{\linewidth}{l X}
	\textbf{Operation} & pressed(key : Message) \\
	\hline
	\textbf{Cross References} & UC01 \\
	\hline
	\textbf{Preconditions} & Der Client befindet sich im Spiel und der Countdown ist abgelaufen \\
	\hline
	\textbf{Postconditions} & 
	\begin{minipage}{4in}
		\vskip 4pt
		\begin{itemize}
			\item Der Event ist beim Server angekommen
			\item Der Event befindet sich in der Eventqueue
		\end{itemize}
		\vskip 4pt
	\end{minipage}  \\
\end{tabularx}
\\ \\
\begin{tabularx}{\linewidth}{l X}
	\textbf{Operation} & update(mapChanges : Message) \\
	\hline
	\textbf{Cross References} & UC01 \\
	\hline
	\textbf{Preconditions} & Der Client befindet sich im Spiel \\
	\hline
	\textbf{Postconditions} & 
	\begin{minipage}{4in}
		\vskip 4pt
		\begin{itemize}
			\item Die Informationen über Änderungen in der Map wurden an den Client übertragen
			\item Der Client hat seine Map angepasst
		\end{itemize}
		\vskip 4pt
	\end{minipage}  \\
\end{tabularx}
\\ \\
\begin{tabularx}{\linewidth}{l X}
	\textbf{Operation} & results(results : Array) \\
	\hline
	\textbf{Cross References} & UC01 \\
	\hline
	\textbf{Preconditions} & Die Runde ist vorbei \\
	\hline
	\textbf{Postconditions} & 
	\begin{minipage}{4.8in}
		\vskip 4pt
		\begin{itemize}
			\item Die Clients haben die Rangliste erhalten
			\item Die Clients zeigen die Rangliste an
		\end{itemize}
		\vskip 4pt
	\end{minipage}  \\
\end{tabularx}
\\ \\
\begin{tabularx}{\linewidth}{l X}
	\textbf{Operation} & endOfGame() \\
	\hline
	\textbf{Cross References} & UC01 \\
	\hline
	\textbf{Preconditions} & Alle Runden sind vorbei \\
	\hline
	\textbf{Postconditions} & 
	\begin{minipage}{4in}
		\vskip 4pt
		\begin{itemize}
			\item Alle Clients wurden über das Ende des Spiels informiert
			\item Die Clients zeigen die Rangliste an
		\end{itemize}
		\vskip 4pt
	\end{minipage}  \\
\end{tabularx}
\\ \\
\begin{tabularx}{\linewidth}{l X}
	\textbf{Operation} & disconnect() \\
	\hline
	\textbf{Cross References} & UC01 \\
	\hline
	\textbf{Preconditions} & Der Client befindet sich in der Lobby \\
	\hline
	\textbf{Postconditions} & 
	\begin{minipage}{4in}
		\vskip 4pt
		\begin{itemize}
			\item Der Client ist nicht mehr mit dem Server verbunden
			\item Der Server hat den Client aus der Clientliste entfernt
		\end{itemize}
		\vskip 4pt
	\end{minipage}  \\
\end{tabularx}

\end{document}