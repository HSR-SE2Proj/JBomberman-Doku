\documentclass[11pt]{scrartcl}

\title{Schlussbericht: JBomberman}
\author{Silvan Adrian \\ Fabian Binna \\ Pascal Kistler}
\date{\today{}}

\usepackage[ngerman]{babel}
\usepackage[automark]{scrpage2}
\usepackage{hyperref}
\usepackage{color}
\usepackage[normalem]{ulem}
\usepackage{scrpage2}
\usepackage{graphicx}
\usepackage{tabularx}
\graphicspath{ {./images/} }
\pagestyle{scrheadings}

\clearscrheadfoot
\ihead{\includegraphics[scale=0.4]{jbomberman}}
\ohead{Projekt: JBomberman}
\ifoot{Schlussbericht: JBomberman}
\cfoot{Version: 1.04}
\ofoot{Datum: 28.05.15}
\setheadsepline{0.5pt}
\setfootsepline{0.5pt}

\usepackage{ucs}
\usepackage[utf8x]{inputenc}
\usepackage[T1]{fontenc}


\begin{document}
\def\arraystretch{1.5}
\begin{titlepage}
\begin{center}
\vspace{10em}
\includegraphics[scale=2]{jbomberman}
\vspace{10em}
\end{center}
\begin{center}
\huge {Projekt: JBomberman} \\
\huge {Schlussbericht}
\end{center}
\begin{center}
\vspace{10em}
\LARGE {Pascal Kistler} \\
\LARGE {Silvan Adrian} \\
\LARGE {Fabian Binna}
\end{center}

\end{titlepage}

\newpage
\section{Änderungshistorie}
\label{sec:Änderungen}

\begin{tabularx}{\linewidth}{l l X l}
\textbf{Datum} & \textbf{Version} & \textbf{Änderung}  & \textbf{Autor} \\
\hline
\textbf{09.03.15} & 1.00 & Erstellung des Dokuments & Gruppe \\
\textbf{27.05.15} & 1.01 & Zielerreichung + Allg. Erfahrungsbericht & Silvan 
Adrian\\
\textbf{27.05.15} & 1.02 & Eigener Erfahrungsbericht hinzugefügt & Silvan 
Adrian\\
\textbf{27.05.15} & 1.03 & Eigener Erfahrungsbericht hinzugefügt & Fabian Binna\\
\textbf{28.05.15} & 1.04 & Eigener Erfahrungsbericht hinzugefügt & Pascal 
Kistler\\
\end{tabularx}

\newpage
\tableofcontents
\newpage

\section{Zielerreichung}
Ziel war es das Spiel JBomberman lauffähig über mehrere Runden spielen zu 
können, dies haben wir auch erreicht.
Zudem war es uns wichtig mehr Erfahrungen mit den Entwicklungstools zu sammeln,
von STAN über GIT und Jenkins und so auch die Möglichkeiten 
in einem grösseren Team zu arbeiten auszutesten und erste Erfahrungen 
für folgende Arbeiten zu bekommen.

\section{Allgemeiner Erfahrungsbericht}
Schlechte Erfahrungen haben wir hauptsächlich damit gemacht, 
das man sich besser auf das Projekt vorbereiten sollte.
Dies geht über Spezifikationen zu Commits, 
Tickets im Redmine , Qualitätsmassnahmen etc.
Dies lief bei uns teilweise ziemlich aus dem Ruder, 
da nicht von Anfang an alles
 festgelegt war und wir im Vorhinein keine Erfahrung damit hatten, ob es 
 überhaupt eine gute Entscheidung ist so vorzugehen.
 Gute Erfahrung hatten wir allerdings im Team gemacht, da wir das erste Mal
 in einem grösseren Team arbeiten konnten für ein grösseres Softwareprojekt
 und einen ersten Einblick bekommen haben, wie es vor sich geht.
 Meetings waren bei uns vielmals nicht nötig und es wurde viel
 mündlich abgesprochen, was auch zu einem Problem werden konnte,da
 zwar alle wichtigen Punkte besprochen, aber nicht aufgeschrieben wurden.
 Schlussendlich hat es aber ziemlich gut funktioniert und grössere Fehler
 sind uns zum Glück erspart geblieben, jedenfalls sind wir mit dem Endprodukt
  zufrieden.
  \newpage
\section{Persönliche Erfahrungen}

\subsection{Silvan Adrian}
Ganz am Anfang des Semesters hab ich mir noch lange überlegt, was denn nun ein
passendes Projekt wäre um als SE2Projekt durchzuführen.
Dabei hatte ich Ideen von Online Kochbuch bis Open Source Fitnesstracker 
 oder die Idee einer art Reiseplaner im Web (hatten wir 
bereits als Buisness Plan im Modul BuRe1 gemacht, daher wären 
alle Infos schon vorhanden gewesen).
Schlussendlich wollten meine Teamkameraden nicht viel davon wissen und
wenn möglich sollte das Projekt so einfach, wie möglich sein, jedoch wurde auch 
Sokoban abgelehnt, aber ab da war klar das es ein Spiel werden sollte.
Ich hatte zumindest keine Ahnung von Spielentwicklung, noch sehr viel Erfahrung in 
der Softwareentwicklung, daher war ich jedenfalls froh das Fabian bereits schon mal was in
der Richtung gemacht hatte.
Nach anfänglich Schwierigkeiten (passiert halt wenn man nicht zur 
Beratung geht und keine Erfahrung hat), lief es danach eigentlich ganz gut.
Ich war neben der Entwicklung auch noch für die Serverumgebung zuständig, dabei
bekamen wir einen Server von der HSR als Build Server.
Da ich zuvor noch nie in einer solchen Umgebung gearbeitet habe, brauchte es 
eine Weile bis alles nach Plan lief.
Zudem gab es immer wieder mal Probleme mit Ant, da wir 2 einzelne Applikationen
(Client und Server) aus dem gleichen Workspace buildeten. 
Jedoch hatten wir teilweise Abhängigkeiten zwischen dem Client und dem Server 
package was wir schlussendlich aber dann noch gelöst hatten.
Ein Höhepunkt meiner Meinung nach waren die 2 Usabilitytests, die 
wir durchgeführt hatten welche uns ziemlich schnell aufzeigten, wo noch Probleme
bestanden haben, welche wir wohl durch eigenes testen nicht so schnell gefunden 
hätten.
Alles in allem fand ich es ein gelungenes Projekt, das Endprodukt gefällt mir 
selbst jedenfalls ganz gut.
\newpage
\subsection{Fabian Binna}
Das SE2-Projekt war für mich das erste Softwareprojekt. Es war teilweise schwierig für mich bei der Architektur die richtigen Entscheidungen zu treffen. Ich kannte mich zwar in der Theorie mit dem Themengebiet Computerspiele aus, trotzdem war es nicht leicht eine Spielearchitektur von Grund auf zu entwickeln. Nach dem Projekt würde ich einiges anders machen. Ich würde bessere Technologien wählen und während der Entwicklung genauer auf den geschriebenen Code achten. Der Code wurde teilweise unleserlich.

Wir haben das Projekt zwar gut geplant und bei der Risikoanalyse alle Probleme eliminiert, dafür hatten wir aber Probleme mit der Versionskontrolle und den Tickets. Bei der Versionskontrolle konnten wir uns auf kein gescheites Vorgehen einigen und mussten während dem Projekt zweimal die Strategie ändern. Bei den Tickets hat es manchmal ein mittelgrosses Chaos gegeben, weil wir nicht klar definiert haben wie die Tickets aussehen müssen.

Die Arbeit im Team verlief gut. Da jeder andere Stärken mitbrachte wurden die Aufgaben klar verteilt und nur bei grossen Entscheidungsfragen hat das ganze Team an einer Aufgabe gearbeitet. Es wurden alle Schritte vorher im Team besprochen, so wusste man immer genau wo das Projekt stand.

Beim nächsten Projekt würde ich auf jedenfall mehr Technologien evaluieren, und das Projektmanagement genauer definieren. Wie sehen Tickets aus, wie sieht die Versionskontrolle aus.
\newpage
\subsection{Pascal Kistler}
Für mich war dies das erste Projekt an dem ich von A-Z mit geplant, implementiert und getestet habe. In der Vergangenheit habe ich mehrere Projekte zusammen mit Kollegen angepackt, meistens aus spontaner Lust wieder einmal etwas umzusetzen. Vielfach mangelte es dann an Planung und Durchhaltewillen, sodass selten ein fertiges Produkt daraus entstanden ist. Daneben sind über die Zeit auch ein paar kommandozeilenbasierte Games entstanden, jedoch nie etwas Grösseres, was eine durchdachte Architektur voraussetzt.

Da ich selbst noch nie ein solches Game programmiert habe, hatte ich wenig Ahnung wie man am besten eine solche Gamearchitektur designed. In diesem Bereich habe ich sicher sehr profitiert. Durch das Projekt lernte ich viele  verschiedenen Tools und Technologien kennen, welche ich zuvor nur aus Vorlesungen oder Übungen kannte.

Allgemein verlief das Projekt relativ reibungslos, da hatte ich vor allem zu Beginn mehr Stolpersteine erwartet. 
Schlussendlich bin ich sehr glücklich darüber, dass letztlich ein funktionierendes Game herausgekommen ist.

\end{document}