\documentclass[11pt]{scrartcl}

\title{Schlussbericht: JBomberman}
\author{Silvan Adrian \\ Fabian Binna \\ Pascal Kistler}
\date{\today{}}

\usepackage[ngerman]{babel}
\usepackage[automark]{scrpage2}
\usepackage{hyperref}
\usepackage{color}
\usepackage[normalem]{ulem}
\usepackage{scrpage2}
\usepackage{graphicx}
\usepackage{tabularx}
\graphicspath{ {./images/} }
\pagestyle{scrheadings}

\clearscrheadfoot
\ihead{\includegraphics[scale=0.4]{jbomberman}}
\ohead{Projekt: JBomberman}
\ifoot{Schlussbericht: JBomberman}
\cfoot{Version: 1.01}
\ofoot{Datum: 27.05.15}
\setheadsepline{0.5pt}
\setfootsepline{0.5pt}

\usepackage{ucs}
\usepackage[utf8]{inputenc}
\usepackage[T1]{fontenc}


\begin{document}
\def\arraystretch{1.5}
\begin{titlepage}
\begin{center}
\vspace{10em}
\includegraphics[scale=2]{jbomberman}
\vspace{10em}
\end{center}
\begin{center}
\huge {Projekt: JBomberman} \\
\huge {Schlussbericht}
\end{center}
\begin{center}
\vspace{10em}
\LARGE {Pascal Kistler} \\
\LARGE {Silvan Adrian} \\
\LARGE {Fabian Binna}
\end{center}

\end{titlepage}

\newpage
\section{Änderungshistorie}
\label{sec:Änderungen}

\begin{tabularx}{\linewidth}{l l l l}
\textbf{Datum} & \textbf{Version} & \textbf{Änderung}  & \textbf{Autor} \\
\hline
\textbf{09.03.15} & 1.00 & Erstellung des Dokuments & Gruppe \\
\textbf{27.05.15} & 1.01 & Zielerreichung & Allg. Erfahrungsbericht & Silvan 
Adrian\\
\end{tabularx}

\newpage
\tableofcontents
\newpage

\section{Zielerreichung}
Ziel war es das Spiel JBomberman lauffähig über mehrere Runden spielen zu 
können, dies haben wir auch erreicht.
Zudem war es uns wichtig mehr Erfahrungen mit den Entwicklungstools zu sammeln,
von STAN über GIT und Jenkins und so auch die Möglichkeiten 
in einem grösseren Team zu arbeiten auszutesten und erste Erfahrungen 
für folgende Arbeiten zu bekommen.

\section{Allgemeiner Erfahrungsbericht}
Schlechte Erfahrungen haben wir hauptsächlich damit gemacht, 
das man sich besser auf das Projekt vorbereiten sollte.
Dies geht über Spezifikationen zu Commits, 
Tickets im Redmine erfassen, Qualitätsmassnahmen etc.
Dies lief bei uns teilweise ziemlich aus dem Ruder, 
da nicht von Anfang an alles
 festgelegt war und wir im Vorhinein keine Erfahrung damit hatten, ob es 
 überhaupt eine gute Entscheidung ist so vorzugehen.
 Gute Erfahrung hatten wir allerdings im Team gemacht, da wir das erste Mal
 in einem grösseren Team arbeiten konnten für ein grösseres Softwareprojekt
 und einen ersten Einblick bekommen haben, wie es vor sich geht.
 Meetings waren bei uns vielmals nicht nötig und es wurde viel
 mündlich abgesprochen, was auch zu einem Problem werden konnte,da
 zwar alle wichtigen Punkte besprochen aber nicht aufgeschrieben wurden.
 Schlussendlich hat es aber ziemlich gut funktioniert und grössere Fehler
 sind uns zum Glück erspart geblieben, jedenfalls sind wir mit dem Endprodukt
  zufrieden.
\section{Persönliche Erfahrungen}

\subsection{Silvan Adrian}
<Die Erfahrungen eines Teammitgliedes während der gesamten Projektdauer>
\subsection{Fabian Binna}

\subsection{Pascal Kistler}

\end{document}