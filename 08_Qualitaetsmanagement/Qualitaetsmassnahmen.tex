\documentclass[11pt]{scrartcl}

\title{Qualitätsmassnahmen}
\author{Silvan Adrian \\ Fabian Binna \\ Pascal Kistler}
\date{\today{}}

\usepackage[ngerman]{babel}
\usepackage[automark]{scrpage2}
\usepackage{hyperref}
\usepackage{color}
\usepackage[normalem]{ulem}
\usepackage{scrpage2}
\usepackage{graphicx}
\usepackage{tabularx}
\graphicspath{ {./images/} }
\pagestyle{scrheadings}

\clearscrheadfoot
\ihead{\includegraphics[scale=0.4]{jbomberman}}
\ohead{Projekt: JBomberman}
\ifoot{Qualitaetsmassnahmen}
\cfoot{Version: 1.04}
\ofoot{Datum: 27.05.15}
\setheadsepline{0.5pt}
\setfootsepline{0.5pt}

\usepackage{ucs}
\usepackage[utf8]{inputenc}
\usepackage[T1]{fontenc}


\begin{document}
\def\arraystretch{1.5}
\begin{titlepage}
\begin{center}
\vspace{10em}
\includegraphics[scale=2]{jbomberman}
\vspace{10em}
\end{center}
\begin{center}
\huge {Projekt: JBomberman} \\
\huge {Qualitätsmassnahmen}
\end{center}
\begin{center}
\vspace{10em}
\LARGE {Pascal Kistler} \\
\LARGE {Silvan Adrian} \\
\LARGE {Fabian Binna}
\end{center}

\end{titlepage}

\newpage
\section{Änderungshistorie}
\label{sec:Änderungen}

\begin{tabularx}{\linewidth}{l l l l}
\textbf{Datum} & \textbf{Version} & \textbf{Änderung}  & \textbf{Autor} \\
\hline
\textbf{13.04.15} & 1.00 & Erstellung des Dokuments & Gruppe \\
\textbf{13.04.15} & 1.01 & Qualitätsmassnahmen & Pascal Kistler \\
\bf{28.04.15} & 1.02 & Massnahmen angepasst & Silvan Adrian\\
\bf{18.05.15} & 1.03 & GitLab hinzugefügt + Codreview angepasst & Silvan Adrian \\
\bf{27.05.15} & 1.04 & Vorbereitung Abgabe & Silvan Adrian\\
\end{tabularx}

\newpage
\tableofcontents
\newpage


\section{Massnahmen zur Qualitätssicherung}
\subsection{Git Repository}
Über das Git Repository haben alle Zugriff auf den aktuellen Code.
Jedes Teammitglied erstellt eigene branches für die Features.
Sobald ein Feature fertig ist kann dies in den Masterbranch gemerged werden, 
sodass im Master immer eine voll funktionsfähige Version  ist.

Die Dokumente der Dokumentation werden in einem separaten 
Repository versioniert. In diesem Repository werden keine Branches 
erstellt, sondern alles in den Master gepushed.
\subsection{JUnit Tests}
JUnit-Tests werden verwendet, um kleinere Einheiten der Software 
automatisiert auf Fehler zu überprüfen. Diese werden vor jedem Commit ausgeführt, 
so dass kein Code mit fehlgeschlagenen Tests eingecheckt wird.
\subsection{Findbugs}
Findbugs wird eingesetzt, um durch statische Codeanalyse Bugs 
oder mögliche Probleme im Code zu finden. FindBugs liefert dem Entwickler 
Hinweise auf mögliche oder tatsächliche Probleme im Code.
\subsection{Jacoco}
Jacoco wird eingesetzt um die Test Code Coverage zu sehen und zu verbessern.
\subsection{Jenkins Buildserver}
Nach jedem Commit startet der Jenkins Server einen Build vom Masterbranch. 
Dabei führt Jenkins alle JUnit-Tests aus und sucht mittels FindBugs nach
 Fehlern im Code. Sollten nicht alle Tests erfolgreich sein muss der 
 Fehler sofort behoben werden. Ein Merge mit dem Master wird nur durchgeführt, 
 wenn alle Tests grün sind.
 Zusätzlich zum Findbugs wird mit Jacoco eine Test Code Coverage Analyse 
 durchgeführt.
 \subsection{GitLab}
 Bei jedem Push oder merge in das Code repository kriegt jedes Teammitglied eine 
 Notification über Slack, welche es jedem Teammitglied erlaubt sofort die 
 neusten Änderungen anzusehen und womöglich einen Kommentar dazu abzugeben.
\subsection{CodeReview}
Zum Schluss des Projekts hin wird ein Codereview im Team durchgeführt, um 
nochmals den ganzen Code durchzugehen.



\end{document}