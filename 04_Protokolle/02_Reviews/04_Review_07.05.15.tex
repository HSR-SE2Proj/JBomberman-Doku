\documentclass[11pt]{scrartcl}

\title{Review vom 07.05.2015}
\author{Silvan Adrian \\ Fabian Binna \\ Pascal Kistler}
\date{\today{}}

\usepackage[ngerman]{babel}
\usepackage[automark]{scrpage2}
\usepackage{hyperref}
\usepackage{color}
\usepackage[normalem]{ulem}
\usepackage{scrpage2}
\usepackage{tabularx}
\pagestyle{scrheadings}

\clearscrheadfoot
\ihead{Reviewprotokoll}
\ofoot{Datum: 07.05.15}
\setheadsepline{0.5pt}
\setfootsepline{0.5pt}

\usepackage{ucs}
\usepackage[utf8]{inputenc}
\usepackage[T1]{fontenc}


\begin{document}

{\huge Review Meilenstein 4 vom 7.Mai 2015}

\section{Allgemeines}
\label{sec:Allgemein}

\subsection{Anwesend}
\label{sec:Anwesend}
\textbf{Projektmitglieder:} Silvan Adrian, Fabian Binna, Pascal Kistler \\
\textbf{Projektbetreuer:} Daniel Keller
\section{Input}
\subsection{Code}
\begin{itemize}
 \item Logging anstatt Ausgaben auf Konsole
 \item Random Methode für Random (anstatt 120 direkt mit 100 arbeiten für Wahr) (besser formulieren, besser lesbar -> z.B.: probabiltyGreaterThen)
 \item Kommentare Leerzeilen etc. rausnehmen und löschen
 \item Wiederholender Code Refactoring
 \item Code Conventions einhalten
 \item Aufräumen nicht vergessen (ServerGame)
\end{itemize}

\subsection{Dokument}
\begin{itemize}
  \item Rabbitmq als Package einfügen
  \item Timer (nicht richtig platziert) (STAN mal noch laufen lassen) 
  \item -> Wenn möglich mal noch ein CodeReview zu machen
  \newline
  \textbf{Domain/Game:} 
  \item ServerGame als Creator einzeichnen und 
  nicht nur Interface etc.)
  \newline
  \textbf{Architektur Client:}
  \item GameLoop im Client Game -> aufräumen
  \item Interface auf eine Ebene mit Party, ActionQueue
\end{itemize}

\subsection{Unit Testing/Testing}
\begin{itemize}
   \item Collisions
   \item Timing, Aufnahme PowerUp
\end{itemize}


\section{Benotung}
\label{sec:Benotung}
Note: \underline {5.5}

\end{document}