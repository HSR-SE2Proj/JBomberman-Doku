\documentclass[11pt]{scrartcl}

\title{Review vom 09.04.2015}
\author{Silvan Adrian \\ Fabian Binna \\ Pascal Kistler}
\date{\today{}}

\usepackage[ngerman]{babel}
\usepackage[automark]{scrpage2}
\usepackage{hyperref}
\usepackage{color}
\usepackage[normalem]{ulem}
\usepackage{scrpage2}
\usepackage{tabularx}
\pagestyle{scrheadings}

\clearscrheadfoot
\ihead{Reviewprotokoll}
\ofoot{Datum: 09.04.15}
\setheadsepline{0.5pt}
\setfootsepline{0.5pt}

\usepackage{ucs}
\usepackage[utf8]{inputenc}
\usepackage[T1]{fontenc}


\begin{document}

{\huge Review Meilenstein 3 vom 9.April 2015}

\section{Allgemeines}
\label{sec:Allgemein}

\subsection{Anwesend}
\label{sec:Anwesend}
\textbf{Projektmitglieder:} Silvan Adrian, Fabian Binna, Pascal Kistler \\
\textbf{Projektbetreuer:} Daniel Keller
\section{Input}
\subsection{Generell}
\begin{itemize}
  \item Aufschreiben -> textlich (Austausch von messages -> 
  async Diagramm auch möglich (SSD))
-> Was passiert bei Tastendruck vom Client zum Server
  \item Besprechungen/Erkenntnisse immer aufschreiben
\end{itemize}

\subsection{ArchitekturDokument}
\begin{itemize}
  \item Naming Party -> Lobby
\end{itemize}

\subsection{Tests}
\begin{itemize}
   \item Mindestens Unit Tests vorhanden
\end{itemize}

\subsection{Risikomanagement}
\begin{itemize}
  \item Kommentar anfügen, wie genau umgangen
\end{itemize}

\subsection{Projektplanung}
\begin{itemize}
  \item Construction Phase (Features Tickets, Usability Tests, Beta Tests)
\end{itemize}

\subsection{Zusatz}
\begin{itemize}
  \item Termin nächste Woche abmachen nochmals abklären und verbessern.
\end{itemize}

\section{Benotung}
\label{sec:Benotung}
Note: \underline {4.5}

\end{document}