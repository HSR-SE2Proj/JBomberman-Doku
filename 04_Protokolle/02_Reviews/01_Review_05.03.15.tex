\documentclass[11pt]{scrartcl}

\title{Review vom 05.03.2015}
\author{Silvan Adrian \\ Fabian Binna \\ Pascal Kistler}
\date{\today{}}

\usepackage[ngerman]{babel}
\usepackage[automark]{scrpage2}
\usepackage{hyperref}
\usepackage{color}
\usepackage[normalem]{ulem}
\usepackage{scrpage2}
\usepackage{tabularx}
\pagestyle{scrheadings}

\clearscrheadfoot
\ihead{Reviewprotokoll}
\ofoot{Datum: 05.03.15}
\setheadsepline{0.5pt}
\setfootsepline{0.5pt}

\usepackage{ucs}
\usepackage[utf8]{inputenc}
\usepackage[T1]{fontenc}


\begin{document}

{\huge Review Meilenstein 1 vom 5.März 2015}

\section{Allgemeines}
\label{sec:Allgemein}

\subsection{Anwesend}
\label{sec:Anwesend}
\textbf{Projektmitglieder:} Silvan Adrian, Fabian Binna, Pascal Kistler \\
\textbf{Projektbetreuer:} Daniel Keller, Andreas Steffen
\section{Input}
\label{sec:Input}
\subsection{Entwurfsskizzen}
\label{sec:Entwurfsskizzen}
\begin{itemize}
      \item Domain Skizze
	 \item Use Cases Skizze (max 30 min pro Use Case) -> Oder einfach längeres Spielprinzip schreiben, da keine Use Cases vorhanden)
	\item Architektur Skizze
	\item Nicht funktionale Anforderungen
	\item GUI Entwürfe
\end{itemize}

\subsection{Projektplan}
\label{sec:Projektplan}
\begin{itemize}
      \item Nicht generisch schreiben
	\item Mehr Ergebnisse ( Feste Ergebnisse -> wirkliche Ergebnisse aufschreiben -> keine Use Cases)
	\item Code Analysis ( Checkstyle etc.)
\end{itemize}

\subsection{Testing}
\label{sec:Testing}
\begin{itemize}
      \item Definition of Done (wann wird was eingecheckt)
	\item Wann genau Code Review (nicht nach jedem Commit)
\end{itemize}

\subsection{Versioning}
\label{sec:Versioning}
\begin{itemize}
      \item Dokumentation Zwischenstände (Versionsstände mit PDF -> nicht im GIT Repository sondern 	irgendwo anders gespeichert)
      \item LaTex einsetzen?!
\end{itemize}

\subsection{Risiken}
\label{sec:Risiken}
\begin{itemize}
      \item Wann genau werden die Risiken angepackt
	\item Arbeitspakete einplanen (wie geht man das Risiko an)
	\item -> Risiken abgedeckt (muss ersichtlich sein -> wirkliches Vorgehen)
	\item Libraries Risikos (testen ob so möglich -> als Risiko Bsp.: Java 2D)
	\item Zu generisch -> noch expliziter nennen (Bsp.: Nicht vorhandenes Know-How)
	\item Third Party Libraries
\end{itemize}

\section{Benotung}
\label{sec:Benotung}
Note: \underline {4.5}

\end{document}