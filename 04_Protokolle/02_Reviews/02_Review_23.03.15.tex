\documentclass[11pt]{scrartcl}

\title{Review vom 23.03.2015}
\author{Silvan Adrian \\ Fabian Binna \\ Pascal Kistler}
\date{\today{}}

\usepackage[ngerman]{babel}
\usepackage[automark]{scrpage2}
\usepackage{hyperref}
\usepackage{color}
\usepackage[normalem]{ulem}
\usepackage{scrpage2}
\usepackage{tabularx}
\pagestyle{scrheadings}

\clearscrheadfoot
\ihead{Reviewprotokoll}
\ofoot{Datum: 23.03.15}
\setheadsepline{0.5pt}
\setfootsepline{0.5pt}

\usepackage{ucs}
\usepackage[utf8]{inputenc}
\usepackage[T1]{fontenc}


\begin{document}

{\huge Review Meilenstein 2 vom 23.März 2015}

\section{Allgemeines}
\label{sec:Allgemein}

\subsection{Anwesend}
\label{sec:Anwesend}
\textbf{Projektmitglieder:} Silvan Adrian, Fabian Binna, Pascal Kistler \\
\textbf{Projektbetreuer:} Daniel Keller
\section{Input}
\label{sec:Input}
\subsection{Domainmodell}
\label{sec:Domainmodell}
\begin{itemize}
    \item Verbindung zwischen Bombern und JBomberman beschreiben (kann 8 Bomben tragen)
    \item Wenn Domainmodell schon zu aussagekräftig (Übersetzung von Englischen Begriffen direkt ins Deutsche) -> Im Spielprinzip festlegen -> englische Begriffe einfügen
    \item PowerUp -> keine attribute?!
\end{itemize}
\subsection{Konzeptbeschreibung}
\label{sec:Konzeptbeschreibung}
\begin{itemize}
    \item Nicht nötig, da eigentlich im Spielprinzip alles beschrieben ist
\end{itemize}
\subsection{Use Case Diagramm}
\begin{itemize}
    \item Unterteilung Systemoperationen zwischen Client/Server
\end{itemize}
\subsection{Contracts}
\begin{itemize}
    \item Zu einfach -> kein Bezug aufs Domainmodell
\end{itemize}


\subsection{Anforderungsspezifikation}
\begin{itemize}
    \item Spielprinzip -> OK
    \item Use Case -> OK
    \item Nicht funktionale Anforderungen
    \subitem Leistung nicht nötig, da in Effizienz
    \subitem Runden festlegen oder in Config festgelegt (kurze Beschreibung dazu)
    \item -> abhackbar
\end{itemize}


\section{Benotung}
\label{sec:Benotung}
Note: \underline {5.2}

\end{document}