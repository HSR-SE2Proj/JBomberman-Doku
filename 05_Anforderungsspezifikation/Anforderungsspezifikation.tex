\documentclass[11pt]{scrartcl}

\title{Anforderungsspezifikation}
\author{Silvan Adrian \\ Fabian Binna \\ Pascal Kistler}
\date{\today{}}

\usepackage[ngerman]{babel}
\usepackage[automark]{scrpage2}
\usepackage{hyperref}
\usepackage{color}
\usepackage[normalem]{ulem}
\usepackage{scrpage2}
\usepackage{graphicx}
\usepackage{tabularx}
\usepackage{sidecap}
\graphicspath{ {./images/} }
\pagestyle{scrheadings}
\usepackage{wrapfig}
\usepackage{longtable, tabu}

\clearscrheadfoot
\ihead{\includegraphics[scale=0.4]{jbomberman}}
\ohead{Projekt: JBomberman}
\ifoot{Anforderungsspezifikation}
\cfoot{Version: 1.07}
\ofoot{Datum: \today{}}
\setheadsepline{0.5pt}
\setfootsepline{0.5pt}

\usepackage{ucs}
\usepackage[utf8]{inputenc}
\usepackage[T1]{fontenc}


\begin{document}
\def\arraystretch{1.5}
\begin{titlepage}
\begin{center}
\vspace{10em}
\includegraphics[scale=2]{jbomberman}
\vspace{10em}
\end{center}
\begin{center}
\huge {Projekt: JBomberman} \\
\huge {Anforderungsspezifikation}
\end{center}
\begin{center}
\vspace{10em}
\LARGE {Pascal Kistler} \\
\LARGE {Silvan Adrian} \\
\LARGE {Fabian Binna}
\end{center}

\end{titlepage}

\newpage
\section{Änderungshistorie}
\label{sec:Änderungen}

\begin{tabularx}{\linewidth}{l l l l}
\textbf{Datum} & \textbf{Version} & \textbf{Änderung}  & \textbf{Autor} \\
\hline
\textbf{09.03.15} & 1.00 & Erstellung des Dokuments & Gruppe \\
\textbf{12.03.15} & 1.01 & detailliertes Spielprinzip geschrieben & Silvan Adrian \\
\textbf{13.03.15} & 1.02 & Spielregeln im Detail & Silvan Adrian \\
\textbf{17.03.15} & 1.03 & Allgemeine Beschreibung eingefügt & Silvan Adrian \\
\textbf{17.03.15} & 1.04 & Use Case beschrieben  & Pascal Kistler \\
\textbf{18.03.15} & 1.05 & Bilder ausgetauscht & Silvan Adrian \\
\textbf{19.03.15} & 1.06 & UseCase Diagram hinzugefügt & Pascal Kistler \\
\textbf{20.03.15} & 1.07 & Nichtfunktionale Anforderungen eingefügt & Silvan Adrian \\
\end{tabularx}

\newpage
\tableofcontents
\newpage
\section{Einführung}
\label{sec:Einführung}
\subsection{Zweck}
\label{sec:Zweck}
Dieses Dokument beschreibt die Anforderungen für das Projekt JBomberman.
\subsection{Gültigkeit}
\label{sec:Gültigkeit}
Dieses Dokument ist während des ganzen Projekts gültig und wird laufend aktualisiert.
\subsection{Referenzen}
\label{sec:Referenzen}
siehe letzte Seite Literatur

\subsection{Übersicht}
\label{sec:Übersicht}
In diesem Dokument wird das Produkt hinsichtlich seiner Anforderungen beschrieben, dabei werden Einschränkungen, Annahmen, Abhängigkeiten und Funktionen beschrieben

\section{Allgemein Beschreibung}
\label{sec:Allgemeine Beschreibung}

\subsection{Produkt Perspektive}
\label{sec:Produkt Perspektive}
JBomberman soll ein kostenlos und einfach zu installierendes Game sein, das für alle Desktop Plattformen zur Verfügung steht.
Es soll für bis zu 4 Mitspieler kurzweilige Unterhaltung bieten.
\subsection{Produkt Funktion}
\label{sec:Produkt Funktion}
\begin{itemize}
    \item JBomberman ist ein Klon des Spielklassikers Bomberman
    \item JBomberman ist nur im Mehrspielermodus spielbar
    \item Es gibt einen dedizierten Server, der entweder lokal oder auf einem entfernten Server gestartet werden kann
\end{itemize}

\subsection{Benutzer Chrakteristik}
\label{sec:Benutzer Chrakteristik}
Zur Zielgruppe von JBomberman gehören die, die  sich bereits für den Klassiker Bomberman interessiert haben.

In JBomberman wird vorausgesetzt das der Spieler schon einmal ein Netzwerkfähiges Spiel gespielt hat, da der Spieler selbst einen Server starten und sich über dessen IP verbinden muss.
\subsection{Einschränkungen}
\label{sec:Einschränkungen}
\begin{itemize}
    \item Es können maximal 4 Spieler an einem Spiel teilnehmen, dabei geht das Spiel über mehrere Runden.
    \item Das Spiel soll auf allen gängigen Desktop Plattformen lauffähig sein.
    \item Ein Server kann nur 1 Spiel verwalten, somit benötigt jede Spielaustragung ihren eigenen Server
\end{itemize}
\subsection{Annahmen}
\label{sec:Annahmen}
\begin{itemize}
    \item Spielcomputer verfügt über Tastatur und Maus zur Steuerung
    \item Es wird vorausgesetzt, dass der Spieler mit den Spielregeln vertraut ist
\end{itemize}

\subsection{Abhängigkeit}
\label{sec:Abhängigkeit}
Die lauffähigkeit von JBomberman hat direkte einbussen, wenn folgende Abhängigkeiten nicht erfüllt sind.
\begin{itemize}
    \item Es wird Java in der Version 8 vorausgesetzt
    \item Der Spielcomputer muss sich entweder im gleichen Netzwerk wie der Server befinden oder Zugriff aufs Internet haben
    \item Die Performance hängt von der zur Verfügung stehenden Bandbreite zum Server und der Geschwindigkeit von RabbitMQ beim abarbeiten der Queues ab
\end{itemize}

\newpage
\section{Detailiertes Spielprinzip}
\label{sec:Detailiertes Spielprinzip}
\subsection{Kurzbeschreibung}
\label{Kurzbeschreibung}
In der klassischen Variante besteht das Spielfeld aus einer Anordnung 
von zerstörbaren und unzerstörbaren Wänden. 
Durch das Legen von Bomben können somit immer mehr 
Bereiche des Spielfelds erschlossen werden. Hinter einigen 
Wänden verstecken sich Bonusgegenstände.\cite{Bomberman Spielprinzip}

\subsection{Spielregeln}
\label{sec:Spielregeln}

\subsubsection{Spielewinn}
\label{sec:Spielgewinn}
Ein Spieler hat gewonnen, wenn alle anderen Bombermans besiegt sind und er der letzte bestehende Spieler auf dem Spielfeld ist.
Der Gewinner bekommt einen Punkt für die gewonnene Runde und wird im angerechnet angezeigt in der Statusanzeige.
Der Bomberman mit den meisten gewonnen Runden gewinnt ist schlussendlich der Gewinner.

\subsubsection{Spielabbruch}
\label{sec:Spielabbruch}
Zu einem Spielabbruch kommt es, wenn es zu wenige Mitspieler gibt (mindestens 2 Spieler müssen verbunden sein).
Dies kann eintreffen, falls ein Verbindungsunterbruch zustande kommen sollte.

\subsubsection{Timer läuft ab}
\label{sec:Timer läuft ab}
Falls innerhalb der festgelegten Zeit kein Bomberman alle anderen Mitspieler besiegt wird das Spielfeld innerhalb kurzer Zeit verkleinert.
Die Bomberman, die sich dabei auf einem Block befindet der \grqq{}verschwindet\grqq{} wird aus dem Spiel geworfen.
Der Bomberman, der übrig bleibt wird dabei der Gewinner der Spielrunde.

\subsubsection{Bombe}
\label{sec:Bombe}
Sobald eine Bombe platziert ist wird dessen \grqq{}Zeit bis zur Explosion gestartet\grqq{}.
Sobald ein Bombe explodiert wird alles was sich im Radius der Bombe befindet und zerstörbar ist zerstört.
Der Radius der Bombenexplosion kann varieren, da Powerups für das Erhöhen des Bombenradius bestehen.
\subsubsection{Zerstörbare Wand}
\label{sec:Zerstörbare Wand}
Wenn eine zerstörbare Wand zerstört wird verschwindet diese und macht den Weg frei.
Manche zerstörbaren Gegenstände beherbergen Powerups für die Bombermans, welche nach dem zerstören der Wand aufgedeckt werden.
Solange die Wand aber nicht zerstört ist kann auch kein Bomberman diese durchqueren.

\subsubsection{Unzerstörbare Wand}
\label{sec:Unzerstörbare Wand}
Unzerstörbare Wände können nicht durch Bomben zerstört werden, noch von Bombermans durchquert werden.

\subsubsection{Bomberman}
\label{sec:Bomberman}
Der Bomberman wird über die Tastatur gesteuert, dabei gibt es 4 Tasten für die Fortbewegung und 1 Zusätzliche Taste für das legen der Bombe.
Ein Bomberman kann sich \grqq{}upgraden\grqq{}, wenn er Powerups aufhebt.

\subsection{Spielelemente}
\label{sec:Spielelemente}
\subsubsection{Statusanzeige}
\label{sec:Statusanzeige}
In der Statusanzeige wird für jeden einzelnen Bomberman (1-4 Spieler) die Anzahl 
Gewinne, ein Timer der runterzählt und die momentane Runde angezeigt.
\subsubsection{Spielfeld}
\label{sec:Spielfeld}
Das Spielfeld wird sich auf eine Grösse von ca. 13 Blöcken x 13 Blöcken 
erstrecken.
Die Wände sind dabei unzerstörbar, damit die bis zu 4 Bombermans das Spielfeld 
nicht verlassen können.
Jeder Ecken im Spielfeld ist für einen Bomberman festgelegt, bei weniger als 4 
Spielern bleiben die übrigen Ecken einfach leer.
\begin{center}
Spielfeld in JBomberman mit 4 Spielern.
\includegraphics[scale=1.4]{bombermanmap} 
\end{center}

\newpage
\subsection{Wände}
\label{sec:Wände}
\subsubsection{Unzerstörbar}
\label{sec:Unzerstörbar}
\begin{table}[!h]
\begin{tabularx}{\linewidth}{l X }
     
      \raisebox{-.8\totalheight}{ \includegraphics[scale=.8]{solidblock}}
      & Die unzerstörbaren Wände können von keiner Bombe zerstört noch von einem 
Bomberman durchlaufen werden.
\end{tabularx}
\end{table}


\subsubsection{Zerstörbar}
\label{sec:Zerstörbar}

\begin{table}[!h]
  \begin{tabularx}{\textwidth}{l X }
    
    
       \raisebox{-.8\totalheight}{\includegraphics[scale=0.8]{explodableblock}} & 
       Die zerstörbaren Wände können von 
    einer Bombe zerstört werden, bevor sie zerstört sind kann ein Bomberman 
    diese nicht durchlaufen.
  \end{tabularx}
\end{table}
\subsection{Bomberman}
\label{sec:Bomberman}
\begin{table}[!h]
\begin{tabularx}{\textwidth}{l X}
\raisebox{-.8\totalheight}{\includegraphics[scale=0.6]{bomberman}}
  & Der Bomberman ist der steuerbare Charakter im Bomberman Spiel, dabei kann er 
sich in alle Himmelsrichtungen bewegen.
Es sei denn ein Hindernis ist im Weg, dann kann er sich nicht weiter bewegen.
\end{tabularx}
  
\end{table}



\subsection{Bombe}
\label{sec:Bombe}
\begin{table}[!h]
\begin{tabularx}{\textwidth}{l X}
\raisebox{-.8\totalheight}{\includegraphics[scale=0.8]{bomb}}
& Die Bombe ist wohl der wichtigste Gegenstand im JBomberman, denn ohne die Bombe 
könnten sich die Bombermans ihren Weg nicht freiräumen.
Somit ist die Funktion der Bombe das beseitigen von zerstörbaren Wände und um 
Gegenspieler frühzeitig aus dem Spiel zu werfen.
\end{tabularx}

\end{table}
\newpage
\subsection{Power Ups}
\label{subsec:Power Ups}
Durch zerstören, der zerstörbaren Wände kommen Power Ups zutage, welche einen 
gewissen Einfluss auf das Spielerlebnis haben.

\subsubsection{Bombe}
\label{sec:Bombe}
\begin{table}[!h]
\begin{tabularx}{\textwidth}{l X}
\raisebox{-.6\totalheight}{\includegraphics[scale=0.8]{bombpowerup}}
& Durch dieses Powerup ist es dem Bomberman möglich mehr als eine Bombe zu platzieren, dabei gibt es ein Maxiumum von 8 Bomben welche zur gleichen Zeit gelegt werden können.
\end{tabularx}
\end{table}

\subsubsection{Stiefel}
\label{sec:Stiefel}
\begin{table}[!h]
\begin{tabularx}{\textwidth}{l X}
\raisebox{-.6\totalheight}{\includegraphics[scale=0.8]{speedpowerup}}
& Durch dieses Powerup beschleunigt sich die Fortbewegungsgeschwindigkeit eines Bomberman.
\end{tabularx}

\end{table}

\subsubsection{Flamme}
\label{sec:Flamme}

\begin{table}[!h]
\begin{tabularx}{\textwidth}{l X}
\raisebox{-.6\totalheight}{\includegraphics[scale=0.8]{flamepowerup}}
& Dieses Powerup gibt der Bombe einen grösseren Sprengradius (in alle 4 Himmelsrichtungen).

\end{tabularx}

\end{table}

\newpage
\section{Use Cases}
\label{sec:Use Cases}

\subsection{Use Case Diagramm}
\label{sec:Use Case Diagramm}
\raisebox{-.6\totalheight}{\includegraphics[scale=0.8]{UCDiagram}}

\subsection{Aktoren + Stakeholders}
\label{Aktoren + Stakeholders}
  	\begin{tabularx}{\linewidth}{lll}
  		\bfseries Aktor & \bfseries Typ & \bfseries Ziele \\\hline 
  		Spieler & Primary &  
  		\begin{minipage}{5in}
  			\vskip 4pt
  			\begin{itemize}
  				\item Einfach mit Server verbinden
  				\item Ohne Umwege Spiel starten
  				\item Schnelle Beendigung des Spiels
  				\item Schnelle Reaktionszeit des Spiels
  				\item kurzweilige Unterhaltung
  			\end{itemize}
  			\vskip 4pt
  		\end{minipage}
		\\ \hline
	\end{tabularx}

\newpage
\subsection{Beschreibungen fully dressed}
\label{sec:Beschreibungen full dressd}

\subsubsection{UC01: JBomberman spielen}
\label{sec:UC01: JBomberman spielen}
\belowtabulinesep = 1mm
\begin{longtabu} to \textwidth {X[1,l] X[2,l]}
	\bfseries Primäraktor & Spieler  \\\hline 
	\bfseries Steakholders und Interessen & Spieler: Möchte das Spiel gemeinsam mit Bekannten spielen  \\\hline 
	\bfseries Vorbedingungen & Das Programm wurde gestartet  \\\hline 
	\bfseries Nachbedingungen & Das Programm wurde beendet  \\\hline 
	\bfseries Standartablauf & 
		\begin{enumerate}
			\item Der Spieler gibt die Adresse des Servers ein zu dem er sich verbinden möchte
			\item Spieler drückt auf \textbf{Connect}
			\item Die Verbindung wird hergestellt und andere evtl. wartende Spieler werden in der Lobby angezeigt
			\item Der Spieler klickt \textbf{I'm ready}
			\item Sobald alle Spieler bereit sind, wird das Spiel gestartet
			\item Die Spielumgebung wird gestartet und die erste Runde beginnt
			\item Bis zur letzten Runde wiederholen
			\begin{enumerate}
				\item Der Countdown zu Beginn einer Runde gibt den Spielern Zeit sich und die andern Spieler zu lokalisieren. Nach dem Countdown kontrollieren die Spieler ihre Figur und die Runde beginnt
				\item Am Ende einer Runde wird jedem Spieler angezeigt, ob er gewonnen hat und der Punktestand aller Spieler
			\end{enumerate}
			\item Die Spielumgebung wird beendet und der Spieler befindet sich wieder in der Lobby
			\item Der Spieler verlässt die Lobby indem er auf \textbf{Disconnect} drückt
			\item Der Spieler schliesst das Verbindungsfenster
		\end{enumerate}
\\\hline 
	\bfseries Alternativer Ablauf & 
		\begin{enumerate}
			\setcounter{enumi}{1}
			\item 
			\begin{enumerate}
				\item Der Server ist nicht erreichbar. Das System gibt eine Fehlermeldung aus und der Spieler kann eine andere Adresse angeben (Schritt 1)
			\end{enumerate}
			\item 
			\begin{enumerate}
				\item Der Spieler entscheidet sich um
				\begin{enumerate}
					\item Der Spieler drückt auf \textbf{Disconnect}. 
					\item Die Verbindung wird getrennt und der Spieler befindet sich wieder auf dem Startbildschirm (Schritt 1)
				\end{enumerate}
			\end{enumerate}
			\item 
			\begin{enumerate}
				\item Der Spieler entscheidet sich um
				\begin{enumerate}
					\item Der Spieler drückt auf \textbf{I'm not ready}. 
					\item Solange mindestens ein Spieler nicht bereit ist, startet das Spiel nicht. (Schritt 3)
				\end{enumerate}
			\end{enumerate}
			\setcounter{enumi}{6}
			\item 
			\begin{enumerate}
				\item 
				\begin{enumerate}
					\item Der Spieler schliesst das Fenster
					\item Das Spiel beendet sich und der Startbildschirm erscheint (Schritt 1)
				\end{enumerate}
			\end{enumerate}
		\end{enumerate}  \\\hline 
	\bfseries Spezielle Anforderungen & siehe nichtfunktionale Anforderungen  \\\hline 
	\bfseries Technologie- und Datenvarianten & Keine  \\\hline 
	\bfseries Auftrittshäufigkeit & mehrmals pro Woche  \\\hline 
	\bfseries Offene Fragen & Keine  \\\hline  
\end{longtabu}

\newpage
\section{Nichtfunktionale Anforderungen}
\label{sec:Nicht funktionale Anforderungen}

\subsection{Leistung}
\label{sec:leistung}
\begin{itemize}
    \item Die Software ist zeitkritisch, daher muss auf allen Clients der gleiche Stand vorhanden sein, sonst könnte dies als Vorteil für einen Mitspieler enden
    
\end{itemize}


\subsection{Menge}
\label{sec:Menge}
\begin{itemize}
    \item Anzahl Spieler ist beschränkt (auf 4)
    \item Bombenanzahl ist beschränkt (auf 8 pro Bomberman)
    \item Anzahl Runden werden festgelegt
    \item Spielzeit wird festgelegt (ca.: 3 min)
\end{itemize}



\subsubsection{Schnittstellen}
\label{sec:Schnittstellen}
\begin{itemize}
    \item Zur Intereaktion mit dem Spieler werden herkömmmliche Schnittstellen benötigt (Maus, Tastatur, Bildschirm)
    \item Das System benötigt eine Netzwerkschnittstelle, um mit den anderen Spielern Kommunizieren zu können.
    \item Es wird ein gestartet Server gebraucht, welcher über seine IP erreichbar ist
\end{itemize}

\subsection{Qualitätsmerkmale}
\label{sec:Qualitätsmermale}

\subsubsection{Funktionalität}
\label{sec:Funktionalität}
siehe Detailiertes Spielprinzip


\subsubsection{Zuverlässigkit}
\label{sec:Zuverlässigkeit}
\begin{itemize}
    \item JBomberman bietet nur einen Mehrpsielermodus und dieser soll in 90\% der Fälle durchführbar sein.
    \item Falls ein Spieler die Netzwerkverbindung verliert, wird dieser aus dem laufenden Spiel geworfen.
    \item Die restlichen Spieler können dabei weiterspielen, falles es bereits nur 2 Spieler gab wrd bei beiden das Spiel unterbrochen.
\end{itemize}

\subsubsection{Benutzbarkeit}
\label{sec:Benutzbarkeit}
\begin{itemize}
    \item JBomberman verfügt über ein User Interface, was ein einfaches einsteigen ins Spiel ermöglichen soll
    \item Das Spiel selbst wird nur mit der Tastatur gesteuert
\end{itemize}

\subsubsection{Effizienz}
\label{sec:Effizienz}
\begin{itemize}
    \item Server muss sich 30 mal in der Sekunde updaten und im gleichen Intervall Messages versenden können
    \item Client muss im gleichen Interval alles rendern können
\end{itemize}

\subsubsection{Änderbarkeit}
\label{sec:Änderbarkeit}
Für JBomberman sind vorerst keine Erweiterungen geplant, weitere PowerUps wären allerdings denkbar.


\subsubsection{Übertragbarkeit}
\label{sec:Übertragbarkeit}
Da das Projekt in Java umgesetzt wird ist es Plattformunabhängig, dadurch kann es auch von verschiedenen Spielern auf anderen Plattformen gespielt werden.
In unserem Team selbst sind auch 3 verschiedene Entwicklungsumgebungen vorhanden, wodurch es noch wichtiger ist das die Plattformunabhängigkeit gegeben ist.



\begin{thebibliography}{999}
\bibitem [Spielprinzip] {Bomberman Spielprinzip}
\url{http://de.wikipedia.org/wiki/Bomberman}, Zugriff 
11.03.2015
\end{thebibliography}

\end{document}