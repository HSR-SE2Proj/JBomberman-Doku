\documentclass[11pt]{scrartcl}

\title{Anforderungsspezifikation}
\author{Silvan Adrian \\ Fabian Binna \\ Pascal Kistler}
\date{\today{}}

\usepackage[ngerman]{babel}
\usepackage[automark]{scrpage2}
\usepackage{hyperref}
\usepackage{color}
\usepackage[normalem]{ulem}
\usepackage{scrpage2}
\usepackage{graphicx}
\usepackage{tabularx}
\graphicspath{ {./images/} }
\pagestyle{scrheadings}

\clearscrheadfoot
\ihead{\includegraphics[scale=0.4]{jbomberman}}
\ohead{Projekt: JBomberman}
\ifoot{Anforderungsspezifikation}
\cfoot{Version: 1.00}
\ofoot{Datum: \today{}}
\setheadsepline{0.5pt}
\setfootsepline{0.5pt}

\usepackage{ucs}
\usepackage[utf8]{inputenc}
\usepackage[T1]{fontenc}


\begin{document}
\def\arraystretch{1.5}
\begin{titlepage}
\begin{center}
\vspace{10em}
\includegraphics[scale=2]{jbomberman}
\vspace{10em}
\end{center}
\begin{center}
\huge {Projekt: JBomberman} \\
\huge {Anforderungsspezifikation}
\end{center}
\begin{center}
\vspace{10em}
\LARGE {Pascal Kistler} \\
\LARGE {Silvan Adrian} \\
\LARGE {Fabian Binna}
\end{center}

\end{titlepage}

\newpage
\section{Änderungshistorie}
\label{sec:Änderungen}

\begin{tabularx}{\linewidth}{l l l l}
\textbf{Datum} & \textbf{Version} & \textbf{Änderung}  & \textbf{Autor} \\
\hline
\textbf{09.03.15} & 1.00 & Erstellung des Dokuments & Gruppe \\
\end{tabularx}

\newpage
\tableofcontents
\newpage
\section{Einführung}
\label{sec:Einführung}

\subsection{Gültigkeit}
\label{sec:Gültigkeit}

\subsection{Referenzen}
\label{sec:Referenzen}

\subsection{Übersicht}
\label{sec:Übersicht}

\section{Allgemein Beschreibung}
\label{sec:Allgemeine Beschreibung}

\subsection{Produkt Perspektive}
\label{sec:Produkt Perspektive}

\subsection{Produkt Funktion}
\label{sec:Produkt Funktion}

\subsection{Benutzer Chrakteristik}
\label{sec:Benutzer Chrakteristik}

\subsection{Einschränkungen}
\label{sec:Einschränkungen}

\subsection{Annahmen}
\label{sec:Annahmen}

\subsection{Abhängigkeit}
\label{sec:Abhängigkeit}

\section{Use Cases}
\label{sec:Use Cases}

\subsection{Use Case Diagramm}
\label{sec:Use Case Diagramm}

\subsection{Aktoren & Stakeholders}
\label{Aktoren & Stakeholders}

\subsection{Beschreibungen Brief}
\label{sec:Beschreibungen Brief}

\subsubsection{Use Case Name}
\label{sec:Name of Use Case}

\subsection{Beschreibungen fully dressed}
\label{sec:Beschreibungen full dressed}

\subsubsection{Use Case Name}
\label{sec:Use Case Name}

\section{Weitere Anforderungen}
\label{sec:Weitere Anforderungen}

\subsection{Qualitätsmerkmale}
\label{sec:Qualitätsmermale}

\subsection{Schnittstellen}
\label{sec:Schnittstellen}

\subsection{Randbedingungen}
\label{sec:Randbedingungen}

\end{document}