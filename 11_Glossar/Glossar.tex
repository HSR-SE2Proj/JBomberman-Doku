\documentclass[11pt]{scrartcl}

\title{Glossar: JBomberman}
\author{Silvan Adrian \\ Fabian Binna \\ Pascal Kistler}
\date{\today{}}

\usepackage[ngerman]{babel}
\usepackage[automark]{scrpage2}
\usepackage{hyperref}
\usepackage{color}
\usepackage[normalem]{ulem}
\usepackage{scrpage2}
\usepackage{graphicx}
\usepackage{tabularx}
\graphicspath{ {./images/} }
\pagestyle{scrheadings}

\clearscrheadfoot
\ihead{\includegraphics[scale=0.4]{jbomberman}}
\ohead{Projekt: JBomberman}
\ifoot{Glossar: JBomberman}
\cfoot{Version: 1.07}
\ofoot{Datum: 07.03.15}
\setheadsepline{0.5pt}
\setfootsepline{0.5pt}

\usepackage{ucs}
\usepackage[utf8]{inputenc}
\usepackage[T1]{fontenc}


\begin{document}
\def\arraystretch{1.5}
\begin{titlepage}
\begin{center}
\vspace{10em}
\includegraphics[scale=2]{jbomberman}
\vspace{10em}
\end{center}
\begin{center}
\huge {Projekt: JBomberman} \\
\huge {Glossar}
\end{center}
\begin{center}
\vspace{10em}
\LARGE {Pascal Kistler} \\
\LARGE {Silvan Adrian} \\
\LARGE {Fabian Binna}
\end{center}

\end{titlepage}

\newpage
\section{Änderungshistorie}
\label{sec:Änderungen}

\begin{tabularx}{\linewidth}{l l l l}
\textbf{Datum} & \textbf{Version} & \textbf{Änderung}  & \textbf{Autor} \\
\hline
\textbf{06.04.15} & 1.00 & Erstellung des Dokuments & Gruppe \\
\textbf{06.04.15} & 1.01 & Erste Einträge ins Glossar & Silvan Adrian \\
\end{tabularx}

\newpage
\tableofcontents
\newpage
\section{Glossar}
\subsection{Begriffe}

\begin{tabularx}{\linewidth}{l | X}
    \textbf{Begriff} & \textbf{Beschreibung}\\
    \hline
    JBomberman & Setzt sich aus den Begriffen Java und Bomberman zusammen und ist der Titel des Softwareprojekts\\
    \hline
    GitLab & Eine einfache Weboberfläche zur Verwaltung von Git repositories.\\
    \hline
    RabbitMQ & Service der auf dem Java Messaging Service aufbaut und wird zum Austausch von Messages über Queues gebraucht.\\
    \hline
    Sprite & Eine Grafikdatei, die ins Spiel eingebunden wird.\\
    \hline
    Animated Sprite & Grafikdatei mit einer Animation (Bilder nebeneinander Platziert damit durchgeswitcht werden kann und eine Animation ensteht\\
    \hline
    Lobby & Ort an dem alle Spieler zusammenkommen, bevor das Spiel gestartet wird.\\
    
\end{tabularx}

\subsection{Abkürzungen}
\begin{tabularx}{\linewidth}{l | X}
    \textbf{Abkürzung} & \textbf{Beschreibung}\\
    \hline
    GUI & Graphical User Interface (Grafische Oberfläche)\\
    \hline
    CLI & Command Line Interface  , reine Kommandozeile und keine Grafische Oberfläche.\\
    \hline 
    IP & Internet Protocol\\
    \hline
    TCP & Transfer Control Protocol\\
    \hline
    JMS & Java Messaging Service\\
\end{tabularx}

\end{document}